\documentclass{article}

\usepackage[most]{tcolorbox}
\usepackage{physics}
\usepackage{graphicx}
\usepackage{float}
\usepackage{amsmath}
\usepackage{amssymb}


\usepackage[utf8]{inputenc}
\usepackage[a4paper, margin=1in]{geometry} % Controla los márgenes
\usepackage{titling}

\title{Clase 10 }
\author{Manuel Garcia.}
\date{\today}

\renewcommand{\maketitlehooka}{%
  \centering
  \vspace*{0.05cm} % Espacio vertical antes del título
}

\renewcommand{\maketitlehookd}{%
  \vspace*{2cm} % Espacio vertical después de la fecha
}

\newcommand{\caja}[3]{%
  \begin{tcolorbox}[colback=#1!5!white,colframe=#1!25!black,title=#2]
    #3
  \end{tcolorbox}%
}

\begin{document}
\maketitle

\section{Evolucion temporal de los sistemas }
\caja{green}{Evolucion temporal }{
  \begin{gather*}
    \ket{\psi(t)} = e ^ {-i \frac{Ht }{\hbar }}\ket{\psi(t=0 )}\qquad \text{ Si }H \text{ no depende de t }\\
    i \hbar \frac{d  }{d t }\ket{\psi(t) } = H \ket{\psi(t) }
  \end{gather*}
}

Qué sucede cuando tenemos que $ H = \frac{\vec P ^2}{2m } + V(\vec r ) $?
\begin{gather*}
  \bra{\vec r }\ket{\psi(t) } = \psi(t, \vec r )  \\
  \text{Tenemos que }\\
  \bra{\vec r }\ket{\psi} = \psi(\vec r ) \\
  \bra{\vec r }\vec P \ket{\psi } = -i \hbar \vec \grad_r \psi(\vec r )\\
  \bra{\vec r }\vec P ^2\ket{\psi} = - \hbar ^2 \grad ^2 \psi(\vec r )\\
  \text{Donde }\grad ^2 \text{ Es el laplaciano } \grad ^2 = \frac{\partial ^2 }{\partial x ^2} + \frac{\partial ^2 }{\partial y ^2} + \frac{\partial ^2 }{\partial z ^2  } = \vec \grad . \vec \grad \\
  \text{Por lo tanto obtenemos que: }
  i \hbar \frac{\partial  }{\partial t }\psi (t, \vec r ) = - \frac{\hbar ^2}{2m }\grad ^2 \psi(t, \vec r ) + V(\vec r ) \psi(t, \vec r )
\end{gather*}

\caja{red}{Ecuacion de schrodinger \textbf{dependiente} del tiempo }{
  \begin{gather*}
    i \hbar \frac{\partial  }{\partial t }\psi (t, \vec r ) = - \frac{\hbar ^2}{2m }\grad ^2 \psi(t, \vec r ) + V(\vec r ) \psi(t, \vec r ) 
  \end{gather*}
}

Como es dependiente del tiempo podemos proyectar el Hamiltoniano sobre r: 
\begin{gather*}
  H \ket{\psi_n } = E_n \ket{\psi_n }\\
  \bra{\vec r }\frac{\vec P ^2}{2m }\ket{\psi_n } + \bra{\vec r }V(\vec r )\ket{\psi_n } = E_n \bra{\vec r }\ket{\psi _n }\\
  - \frac{\hbar  ^2}{2m }\grad ^2 \psi_n (\vec r ) + V(\vec r) \psi_n(\vec r ) = E_n \psi_n (\vec r )
\end{gather*}

\caja{red}{Ecuacion de Schrodinger \textbf{independiente} del tiempo }{
  \begin{gather*}
    - \frac{\hbar  ^2}{2m }\grad ^2 \psi_n (\vec r ) + V(\vec r) \psi_n(\vec r ) = E_n \psi_n (\vec r )
  \end{gather*}
  Para el caso independiente del tiempo la ecuacion se hace mucho mas facil de resolver y obtenemos que $ \psi  $ es: 
  \begin{gather*}
    \psi(t, \vec r ) = \displaystyle\sum_{n }^{} c_n e ^ {- \frac{i E_n t }{\hbar }}\psi_n(\vec r)  
  \end{gather*}
  Tener en cuenta que $ \frac{\hbar }{2m } $ es un autovalor de la energia.
}
\caja{black}{}{
  El profesor quedó en enviar unas notas sobre la proyeccion en $ \vec P  $ al moodle.
}
Si tenemos $ V = 0 \quad \rightarrow \quad H = \frac{\vec P ^2}{2m } $ y por lo tanto $ [H,\vec P ] = 0  $.
Un estado sin evolucion temporal es: 
\begin{gather*}
  \bra{\vec r }\ket{E- \frac{P ^2}{2m }, \vec P }  = \frac{1}{(2 \pi \hbar ) ^ {3/2 }} e ^ {i \frac{\vec P \vec r }{\hbar }}
\end{gather*}
Con evolucion temporal: 
\begin{gather*}
  \psi _{E, \vec P } (t, \vec r) = \bra{\vec r }\ket{E- \frac{P ^2}{2m }, \vec P, t }  = \frac{1}{(2 \pi \hbar ) ^ {3/2 }} e ^ {i \frac{\vec P \vec r }{\hbar }} e ^ {-i \frac{t }{\hbar } \frac{P ^2}{2m }}
  = \frac{1}{(2 \pi \hbar )^ {3/2 }}e ^ {-i Et + i \frac{\vec P \vec r }{\hbar }}
\end{gather*}
Recordar que $ E = \frac{\vec P ^2}{2m } $.
\caja{red}{Solucion eq. schrodinger dependiente del tiempo }{
  \begin{gather*}
    \psi _{E, \vec P } (t, \vec r) = \bra{\vec r }\ket{E- \frac{P ^2}{2m }, \vec P, t }  = \frac{1}{(2 \pi \hbar ) ^ {3/2 }} e ^ {i \frac{\vec P \vec r }{\hbar }} e ^ {-i \frac{t }{\hbar } \frac{P ^2}{2m }}
    = \frac{1}{(2 \pi \hbar )^ {3/2 }}e ^ {-i Et + i \frac{\vec P \vec r }{\hbar }}
  \end{gather*}
}
\caja{blue}{Ejercicio }{
  Verificar la solucion anterior. Ver si safisface la ec. diferencial de schrodinger dependiente del tiempo.
}

\section{Campo electrico y magnetico }
Onda electromagnetica plana
\begin{gather*}
  \vec A(t,\vec r ) = \vec \epsilon e ^ {i (-\omega t + \vec k . \vec r)} 
\end{gather*}
Donde $ \vec \epsilon $ es el vector unitario de polarizacion $ \vec \epsilon . \vec \epsilon = 1  $. Ademas tenemos que $ \lambda = \frac{2 \pi}{\left|\vec k \right|} $. Pero en la cuantica thenemos que $ \vec P = \hbar \vec k  $. Es decir una onda electormagnetica ya lleva esto y por lo tanto $ \omega = \frac{E }{\hbar } \rightarrow E = \hbar \omega $. Tambien tenemos que $ \omega = c \left|\vec k \right| \quad \rightarrow\quad \frac{E}{\hbar } = c \frac{\left|\vec P \right|}{\hbar }$. En relatividad la energia y el momento están conectados por $ E = \sqrt{(c\vec P )^2 + (mc ^2)^2} \approx m c ^2 + \frac{P ^2}{2m } + ...  $

La energia de una onda electromagnetica en electromagnetismo es $ E = \int_{}^{}dr ^ {3 } (\vec E ^2 + \vec B ^2) $. El foton no intercabia energia con esta ecuacion si no con $ E = \hbar \omega $ y $ \vec P  = \hbar \vec k $.

\section{Particula que oscila en una dimension (oscilador armonico)}
\begin{gather*}
  H = \frac{P ^2}{2m } + \frac{1}{2} k x ^2 
\end{gather*}
Frecuencia angular de oscilacion clasica: $ \omega = \sqrt{\frac{k }{m }}  \rightarrow k = \omega ^2 m $. 
\begin{gather*}
  H = \frac{P ^2 }{2m } + \frac{1}{2} m \omega ^2 x ^2
\end{gather*}
No existe un sistema fisico como este en la naturaleza pero se toma como ejemplo ya que es de gran utilidad para ver el formalismo matematico.

Tenemos que $ [P,\lambda] = -i \hbar $.

Vamos a construir el operador $ a $.

\begin{gather*}
  a = \sqrt{\frac{m \omega}{2 \hbar }} \left(x + i \frac{P }{m \omega}\right) \qquad \qquad a ^ {\dag } = \sqrt{\frac{m \omega}{2 \hbar }} \left(x - i \frac{P }{m \omega}\right)\\
  \text{No es hermitico.}\\
  [a, a ^ {\dag }] = \frac{m \omega}{2 \hbar } \left(- \frac{i }{m \omega }i \hbar  + \frac{i }{m \omega} (-i \hbar )\right) = 1\\
  [a, a ^ {\dag }] = 1
\end{gather*}

Y el operador $ a ^ {\dag }a  $. Como $ (a ^ {\dag } a )^ {\dag } = a ^ {\dag }a  $ este operador es hermitico.
\begin{gather*}
  a ^ {\dag }a  = \frac{H }{\hbar \omega} - \frac{1}{2}
\end{gather*}
Por lo tanto podemos decir que: 
\begin{gather*}
  H = \hbar \omega \left(a ^ {\dag }a + \frac{1}{2}\right)
\end{gather*}
Como $ H  $ y $ a^\dag a  $ son funcion uno del otro esto nos dice que tienen autovalores comunes.
\begin{gather*}
  a ^ {\dag }a = N \quad(\text{Operador}) \\
  N \ket{n } = n \ket{n }\qquad n:\text{real}\\
  H \ket{n } = \hbar \omega (n + \frac{1}{2}) \ket{n }\\
  \text{El conmutador }\\
  [N, a ] = a ^ {\dag }a a - a a ^ {\dag }a = [a ^ {\dag }, a] = -a \\
  [N, a ] = -a\\
  [N, a ] = Na - a N  = - a \\
  \text{Le aplicamos }\ket{n }\\
  Na \ket{n } - a N \ket{n } = - a \ket{n }\\
  Na \ket{n } = (n -1 )a \ket{n }
\end{gather*}

\end{document}
