\documentclass{article}

\usepackage[most]{tcolorbox}
\usepackage{physics}
\usepackage{graphicx}
\usepackage{float}
\usepackage{amsmath}
\usepackage{amssymb}


\usepackage[utf8]{inputenc}
\usepackage[a4paper, margin=1in]{geometry} % Controla los márgenes
\usepackage{titling}

\title{Clase 18 }
\author{Manuel Garcia.}
\date{\today}

\renewcommand{\maketitlehooka}{%
  \centering
  \vspace*{0.05cm} % Espacio vertical antes del título
}

\renewcommand{\maketitlehookd}{%
  \vspace*{2cm} % Espacio vertical después de la fecha
}

\newcommand{\caja}[3]{%
  \begin{tcolorbox}[colback=#1!5!white,colframe=#1!25!black,title=#2]
    #3
  \end{tcolorbox}%
}

\begin{document}
\maketitle

\section{Momento Angular}
\begin{gather*}
  [J ^ {\alpha}, J ^ {\beta}] = i \hbar  \epsilon ^ {\alpha\beta\gamma} J ^ {\gamma} \qquad \qquad \qquad [J ^ {x }, J ^ {y }] = i \hbar J ^ {z } \text{ etc. }\\
  \vec J ^2 = (J ^ {x })^2 + (J^y)^2 + (J^z )^2 \\
  [J^z, \vec J^2] = 0 
\end{gather*}
Notacion provisional: 
\begin{gather*}
  \vec J ^2 \ket{a,b } = a \ket{a,b }\qquad \qquad J^2 \ket{a, b } = b \ket{a,b } 
\end{gather*}
Operador $ J _{\pm }  $
\begin{gather*}
  J _{\pm } = J _x \pm i J _y  \\
  [\vec J^2, J _{\pm } ] = 0
\end{gather*}
\caja{red}{}{
  \begin{gather*}
    [J^z, J _{\pm } ] = \pm \hbar J _{\pm }
  \end{gather*}
}

Si tenemos el estado $ J _{\pm } \ket{a,b } = \ket{\beta _{\pm } } $, qué propiedades tiene? Utilizando que $[\vec J^2, J _{\pm } ] = 0$
\begin{gather*}
  \vec J^2 \ket{\beta _{\pm } } = a \ket{\beta _{\pm} } 
\end{gather*}
Y con $ [J^z, J _{\pm } ] = \pm \hbar J _{\pm } $
\begin{gather*}
  J^z \ket{\beta _{\pm } } = (b\pm \hbar ) \ket{\beta _{\pm } } 
\end{gather*}
Entonces podemos escribir: 
\begin{gather*}
  J ^ {z } J _{+ } \ket{a, b } = ( J_+ J ^z - [J ^ {z }, J _{\pm } ]) \ket{a, b }\\
  = (J_+ J_z + \hbar J _{+ } ) \ket{a, b } = (b + \hbar )  \underset{\ket{\beta_+ }}{J _{+ }\ket{a, b }}
\end{gather*}
Dado a qué limites tiene b? 
\begin{gather*}
  \vec J^2 - (J^z)^2 = (J^x)^2 + (J^y)^2 \\
  \underset{= a- b^2 }{\bra{a, b } \vec J^2 - (J^z)^2 \ket{a,b }} = \bra{a, b } (J^x)^2 + (J^y)^2 \ket{a,b } \geq 0 \\
  a - b^2 \geq 0 \qquad \rightarrow \qquad a \geq b^2 
\end{gather*}
Tenemos que : 
\begin{gather*}
  J _{+ } \ket{a, b _{max }  } = 0  \\
  J _{-  } J _{+ }  \ket{a, b _{min } } = 0 
\end{gather*}

\begin{align*}
  J _{-  } J _{+ } &= (J^x - iJ^y )(J^x + i J^y )\\
              &= (J^x)^2 + (J^2)^2 + i [J^x, J^y ] = \vec J^2 - (J^z)^2 - \hbar J^z - \hbar J^z
\end{align*}

Tenemos que: 
\begin{gather*}
  b _{max } = \frac{m \hbar }{2} \qquad \qquad n = 0,1,2,3,...\\
  b _{max } = \hbar  J \quad \rightarrow \quad J = \frac{b _{max } }{\hbar } = \frac{m}{2}\\
  a = \hbar  ^2 J(J+1)\\
\end{gather*}

\begin{gather*}
  J^z \ket{J,m } = \hbar  m \ket{J,m } \qquad m: -J,-J+1,...,+J\\
  \vec J ^2 \ket{J, m } = \hbar ^2 J(J+1) \ket{J, m }
\end{gather*}

\begin{gather*}
  J = 0 \qquad \rightarrow \qquad \ket{0,0 }\\
  J = \frac{1}{2} \quad \rightarrow \quad \ket{ \frac{1}{2}, - \frac{1}{2}}; \quad \ket{\frac{1 }{2}, + \frac{1}{2}}\\
  J = 1 \quad \rightarrow \quad \ket{1,-1 }, \quad \ket{1, 0 }, \quad \ket{1, 1 }
\end{gather*}

Momento angular \textbf{Orbital }
\begin{gather*}
  \vec L = \vec r \cross \vec p \quad \rightarrow \quad \bra{\vec r } \vec L \ket{\psi} = - i \hbar \vec r \cross \vec \grad \psi(\vec r )
\end{gather*}

\end{document}
