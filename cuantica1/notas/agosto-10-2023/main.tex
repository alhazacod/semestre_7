\documentclass{article}

\usepackage[most]{tcolorbox}
\usepackage{physics}
\usepackage{amsmath}

\usepackage[utf8]{inputenc}
\usepackage[a4paper, margin=1in]{geometry} % Controla los márgenes
\usepackage{titling}

\title{Clase 1}
\author{Manuel Garcia.}
\date{\today}

\renewcommand{\maketitlehooka}{%
  \centering
  \vspace*{0.05cm} % Espacio vertical antes del título
}

\renewcommand{\maketitlehookd}{%
  \vspace*{2cm} % Espacio vertical después de la fecha
}

\newcommand{\caja}[2]{%
  \begin{tcolorbox}[colback=blue!5!white,colframe=blue!25!black,title=#1]
    #2
  \end{tcolorbox}%
}

\begin{document}
\maketitle

\section{Constante de planck}
\caja{$ \hbar  $}{
  \begin{gather}
    \hbar = \frac{h }{2\pi }= 1.055 \times10^{-34}Js\\
    [\hbar ] : Energia *tiempo <-> Accion\\
    E = hf = \hbar \omega \quad \omega : frec. angular.\\
    [\hbar] : momentum \times longitud\\
    N _{av} = 6 \times10^{23}
    \label{eq:hbar }
  \end{gather}
  
}

\section{Ket } % (fold)
\label{sec:Ket }
\caja{ket }{
  \begin{gather}
    \ket{\alpha }-> "vector" n dim. complejo.\\
    c \ket{\alpha} = \ket{\alpha } c\\
    c _{\alpha} + c_\beta = \ket{\delta} \quad lineal.
    \label{eq:Ket }
  \end{gather}
}

Mas adelante de considerara el espacio ket como un vector no numerable e infinito cuyo espacio será llamado espacio de Hilbert.

El espacio dual es la traspuesta conjugada del espacio original. 

\caja{Conjugado de $ \ket{\alpha}:\bra{\alpha} $}{
  \begin{gather}
    \bra{\alpha} = C _{\alpha} ^ {* }\bra{\alpha}+C _{\beta} ^ {* } \bra{\beta}\\
    \bra{\delta}\approx (\delta _{1 } ^ {* }, \delta _{2 } ^ {* }, ..., \delta _{n } ^ {* })\\
    \bra{\alpha} \approx \begin{bmatrix}
        \delta _{1 }  \\
        \delta _{2 }  \\
        ...  \\
        \delta _{n } 
    \end{bmatrix} 
    \label{eq:conjugado }
  \end{gather}
}

Este espacio dual nos sirve para definir un producto punto.
\caja{producto punto }{
  \begin{gather}
    \bra{\alpha}\ket{\delta} = z = \delta _{1 } ^ {* }\alpha _1 + \delta _2 ^ {* }\alpha _{2 } + ... + \delta _{n } ^ {* }\alpha_n\\
    \bra{\beta}\ket{\alpha}=\bra{\alpha}\ket{\beta}^ {* }\\
    \bra{\alpha}\ket{\alpha} = \sum_{n=1 }^{N }|\alpha|^ {2} \geq 0 \\
    \ket{\bar \alpha} = \frac{\ket{\alpha}}{\sqrt{\bra{\alpha}\ket{\alpha}} } \qquad
    \bra{\bar \alpha}\ket{\bar \alpha}=1
    \label{eq:bracket}
  \end{gather}
}

Si $ \bra{\alpha}\ket{\beta} = 0  $ entonces son ortogonales.

\section{operadores} % (fold)
\label{sec:operadores}
\caja{operadores }{
  Para representar cantidades fisicas necesitamos operadores. $ \ket{\alpha} $ es un estado.
  \begin{gather}
    X(\ket{\alpha}) =  X \ket{\alpha} = \ket{\beta}\\
    \bra{\alpha}\ket{\alpha} = 1 \qquad \bra{\beta}\ket{\beta} \neq 1
    \label{eq:norm_ket}\\
    X \ket{\alpha} dc \bra{\alpha}X ^ {\dag}
    \label{eq:operadores }
  \end{gather}
  Si $ X = X ^ {\dag} $ entonces X es hermitico. 
}
\caja{nota }{
  Al comienzo del libro de sakurai se hace la distincion entre el vector normalizado $ \ket{\bar \alpha} $ y el no normalizado $ \ket{\alpha} $, esto solo se hizo en la primera parte del libro ya que desde este punto no se va a hacer la distincion y tendremos que ver en qué caso aplica cada uno. Por esto se hace la distincion de la eq. \ref{eq:norm_ket} donde el vector del espacio ket $ \ket{\alpha} $ está normalizado pero luego de aplicarle el operador deja de estar normalizado.
}

Aquí algunas propiedades de los operadores: 
\begin{gather}
  XY \neq YX \qquad X(YZ) = (XY)Z = XYZ\\
  X(Y \ket{\alpha}) = (XY)\ket{\alpha}=XY \ket{\alpha}, \qquad (\bra{\beta}X)Y = \bra{\beta} (XY) = \bra{\beta}XY\\
  (XY)^ {\dag } = X ^ {\dag } Y ^ {\dag}
  \label{eq:prop.operadores }
\end{gather}
Si para cualquier ket tenemos un operador $X$ tal que $ X \ket{\alpha}=0  $ entonces a este operador lo llamaremos operador nulo.

Los operadores tienen la operacion de adicion con las propiedades asociativas y conmutativas. 
\begin{gather}
  X+Y = Y+X\\
  X+(Y+Z)=(X+Y)+Z
  \label{eq:prop_operadores}
\end{gather}
Existe la escepcion del operador de time-reversal el cual se considerará mas adelante. por ahora todos los operadores son lineales por lo tanto:
\begin{gather}
  \underset{\text{Solo para operadores lineales.}}{X(c_\alpha \ket{\alpha}+c_\beta \ket{\beta}) = c_\alpha X \ket{\alpha}+c_\beta X \ket{\beta}} 
  \label{eq:operadores_lineales}
\end{gather}
% section operadores (end)

\end{document}
