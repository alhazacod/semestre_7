\documentclass{article}

\usepackage[most]{tcolorbox}
\usepackage{physics}
\usepackage{graphicx}
\usepackage{float}
\usepackage{amsmath}
\usepackage{amssymb}


\usepackage[utf8]{inputenc}
\usepackage[a4paper, margin=1in]{geometry} % Controla los márgenes
\usepackage{titling}

\title{Clase 20 }
\author{Manuel Garcia.}
\date{\today}

\renewcommand{\maketitlehooka}{%
  \centering
  \vspace*{0.05cm} % Espacio vertical antes del título
}

\renewcommand{\maketitlehookd}{%
  \vspace*{2cm} % Espacio vertical después de la fecha
}

\newcommand{\caja}[3]{%
  \begin{tcolorbox}[colback=#1!5!white,colframe=#1!25!black,title=#2]
    #3
  \end{tcolorbox}%
}

\begin{document}
\maketitle

\section{}
\begin{gather*}
  H =  \frac{\vec p ^2 }{2m } - \frac{e^2 }{r }\\
  [\vec L , H] = 0 \\
  \rightarrow [L_z, H] = 0 \\
  [\vec L^2 , H ] = 0 
\end{gather*}

Nos dividiremos el spin: 
\begin{gather*}
  H \ket{\psi} = E \ket{\psi} \\
  \ket{\psi} = \ket{E;l,m_e }\\
  H \ket{E;l,m_e } = E \ket{E;l,m_e }\\
  \vec L^2 \ket{E;l,m_e} = \hbar ^2 e(e+1)\ket{E;l,m_e }\\
  L_z \ket{E;l,m_e } = \hbar m_e \ket{E;l,m_e }
\end{gather*}

Queremos hacer la proyeccion: 
\begin{gather*}
  \bra{\underset{= \vec r }{R,\theta, \phi }}\ket{E;l,m_e } =  f(r,\theta,\phi)\\
  L _{ZA } f(r,\theta,\phi) = ?\\
  L _{ZA } = - i \hbar \frac{\partial  }{\partial \phi }
\end{gather*}
Para esto debemos tener en cuenta el momentum angular orbital: 
\begin{gather*}
  \bra{\theta,\phi }\ket{l,m_e } = Y _{l,m_e } (\theta,\phi )  
\end{gather*}
Entonces: 
\begin{gather*}
  L_{ZA}  Y _{l,m_e } (\theta,\phi ) = \hbar  m_e Y _{l,m_e } (\theta,\phi ) \\
  \vec L_A ^2Y _{l,m_e }(\theta,\phi) = \hbar ^2 e(e+1) Y _{l,m_e }(\theta,\phi)\\
  \vec L_A f _{..l,m_e } (r,\theta,\phi) = \hbar ^2 e(e+1) f _{..l,m_e } (r,\theta,\phi)\\
  f _{..l,m_e } (r,\theta,\phi) = \underset{\text{Func. radial}}{R(r)}Y _{l,m_e } (\theta, \phi )
\end{gather*}

\section{Atomo de Hidrogeno } 
\begin{gather*}
  V(r) = -\frac{e^2 }{r } 
\end{gather*}
Su hamiltoniano: 
\begin{gather*}
  H = \frac{\vec p_1 ^2}{2m_e } + \frac{\vec p _2}{2m_p} + V(r) \qquad \qquad \vec r = \vec r_1 - \vec r_2 \qquad r = \left|\vec r \right|
\end{gather*}
Tenemos $ \vec p_1, \vec p_2, \vec r_1, \vec r_2  $
\begin{gather*}
  [p_1^\alpha, r_1^\beta] = -i \hbar  \delta ^ {\alpha \beta }\\
  [p_2^\alpha, r_2^\beta] = -i \hbar  \delta ^ {\alpha \beta }\\
  [p_1^\alpha, r_2^\beta] = 0 \\
  [p_2^\alpha, r_1^\beta] = 0 
\end{gather*}

Posicion y momentum 
\begin{gather*}
  \vec P = \vec p _1 + \vec p _2\qquad  
  \vec R = \frac{m_1 \vec r_1 + m_2 \vec r_2 }{M } 
\end{gather*}
\begin{gather*}
  [p ^ {\alpha }, r ^ {\beta }] = -i \hbar \delta ^ {\alpha, \beta }\\
  \vec p = \frac{m_2 }{M } \vec p _1 - \frac{m_1 }{M } \vec p _2  
\end{gather*}
\begin{gather*}
  [P ^ {\alpha }, R ^ {\beta }] = - i \hbar  \delta ^ {\alpha\beta} \underset{=1 }{\left(\frac{m_1 }{M } + \frac{m_2 }{M }\right) }\\
  [p ^ {\alpha}, r ^ {\beta}] = - i \hbar \delta ^ {\alpha\beta}\\
  [p ^ {\alpha }, R ^ {\beta }] = 0 \\
  [P ^ {\alpha}, r ^ {\beta}] = 0 
\end{gather*}

\hfill 

\begin{gather*}
  H = \frac{\vec p_1 ^ {2 }}{2m_1 }  + \frac{\vec p_2 ^2 }{2m_2 } + V(r)\\
  \text{Reemplazamos: }\qquad 
  \vec p_1 = \vec p + \frac{m_1 }{M }\vec P \qquad 
  \vec p_2 = -\vec p + \frac{m_2 }{M } \vec P \\
  H = \frac{\vec P ^2}{2M } + \frac{\vec p ^2}{2m _{rad } } + V(r) \qquad \text{ con: }\quad \frac{1}{m _{rad } } = \frac{1}{m_1 } + \frac{1}{m_2 }
\end{gather*}

\begin{gather*}
  \bra{\vec R }\ket{\vec P } = \frac{1}{(2\pi \hbar ) ^ {3/2 }} e ^ {i \frac{\vec P \cdot \vec R }{\hbar }} = \psi _{\vec P } (\vec R ) 
\end{gather*}

\begin{gather*}
  H \ket{\vec P ; E_I  }  = \left(\frac{\vec P ^2}{2M } + E_I \right)\ket{\vec P;E_I }
\end{gather*}

\end{document}
