\documentclass{article}

\usepackage[most]{tcolorbox}
\usepackage{physics}
\usepackage{graphicx}
\usepackage{float}
\usepackage{amsmath}
\usepackage{amssymb}


\usepackage[utf8]{inputenc}
\usepackage[a4paper, margin=1in]{geometry} % Controla los márgenes
\usepackage{titling}

\title{Taller \#1}
\author{Manuel Garcia.}
\date{\today}

\renewcommand{\maketitlehooka}{%
  \centering
  \vspace*{0.05cm} % Espacio vertical antes del título
}

\renewcommand{\maketitlehookd}{%
  \vspace*{2cm} % Espacio vertical después de la fecha
}

\newcommand{\caja}[3]{%
  \begin{tcolorbox}[colback=#1!5!white,colframe=#1!25!black,title=#2]
    #3
  \end{tcolorbox}%
}

\begin{document}
\maketitle

\section{}
Se introducen dos estados $ \ket{1 }, \quad \ket{2 } $, ortonormales: 
\begin{gather}
  \bra{1}\ket{1} = \bra{2}\ket{2} = 1 ,\qquad \qquad \bra{1}\ket{2 } = 0 
\end{gather}
Y el estado: 
\begin{gather}
  \ket{\psi } = \frac{1}{\sqrt{3 } }\ket{1 } + \frac{\sqrt{2 } }{\sqrt{3 } }\ket{2 }
\end{gather}
Considerar ahora el estado 
\begin{gather}
  \ket{\phi } = a \ket{1 } + b \ket{2 }
\end{gather}

Con $ a  $ real y positivo y $ b  $ real. 

Determinar $ a  $ y $ b  $ de manera que $ \ket{\phi } $ sea ortogonal a $ \ket{\psi } $, es decir 
\begin{gather}
  \bra{\psi }\ket{\phi  } = 0  
\end{gather}
y tmabien normalizado a 1, es decir 
\begin{gather}
  \bra{\phi }\ket{\phi } = 1  
\end{gather}




\end{document}
