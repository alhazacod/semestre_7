\documentclass{article}

\usepackage[most]{tcolorbox}
\usepackage{physics}
\usepackage{graphicx}
\usepackage{float}
\usepackage{amsmath}
\usepackage{amssymb}


\usepackage[utf8]{inputenc}
\usepackage[a4paper, margin=1in]{geometry} % Controla los márgenes
\usepackage{titling}

\title{Taller \#1}
\author{Manuel Garcia.}
\date{\today}

\renewcommand{\maketitlehooka}{%
  \centering
  \vspace*{0.05cm} % Espacio vertical antes del título
}

\renewcommand{\maketitlehookd}{%
  \vspace*{2cm} % Espacio vertical después de la fecha
}

\newcommand{\caja}[3]{%
  \begin{tcolorbox}[colback=#1!5!white,colframe=#1!25!black,title=#2]
    #3
  \end{tcolorbox}%
}

\begin{document}
\maketitle

\section{}
Se introducen dos estados $ \ket{1 }, \quad \ket{2 } $, ortonormales: 
\begin{gather}
  \bra{1}\ket{1} = \bra{2}\ket{2} = 1 ,\qquad \qquad \bra{1}\ket{2 } = 0 
\end{gather}
Y el estado: 
\begin{gather}
  \ket{\psi } = \frac{1}{\sqrt{3 } }\ket{1 } + \frac{\sqrt{2 } }{\sqrt{3 } }\ket{2 }
\end{gather}
Considerar ahora el estado 
\begin{gather}
  \ket{\phi } = a \ket{1 } + b \ket{2 }
\end{gather}

Con $ a  $ real y positivo y $ b  $ real. 

Determinar $ a  $ y $ b  $ de manera que $ \ket{\phi } $ sea ortogonal a $ \ket{\psi } $, es decir 
\begin{gather}
  \bra{\psi }\ket{\phi  } = 0  
\end{gather}
y tmabien normalizado a 1, es decir 
\begin{gather}
  \bra{\phi }\ket{\phi } = 1  
\end{gather}

\textbf{Solucion: }

Tenemos que: 
\begin{gather*}
  \bra{\psi} = \bra{1 }\frac{1}{\sqrt{3 }  } + \bra{2 }\frac{\sqrt{2 } }{\sqrt{3 } }
\end{gather*}
Por lo tanto: 
\begin{gather*}
  \bra{\psi}\ket{\phi }  = \frac{a }{\sqrt{3 }  } \bra{1}\ket{1} + \frac{\sqrt{2 } b }{\sqrt{3 } }\bra{2}\ket{2}   = 0 
\end{gather*}
Como $ \bra{1}\ket{1 } = 1   $ y $ \bra{2}\ket{2}  =1 $:
\begin{gather*}
   \frac{a }{\sqrt{3 }  }  + \frac{\sqrt{2 } b }{\sqrt{3 } } = 0\\
   a = -\sqrt{2 } b
\end{gather*}
Podemos reescribir $ \ket{\phi } = a \ket{1 } + \frac{a }{\sqrt{2 } \ket{2 }} $ y $ \bra{\phi } = \bra{1 } a + \bra{2 } \frac{a }{\sqrt{2 } } $, entonces: 
\begin{gather*}
  a^2 + \frac{a^2 }{2 } =  1 \quad \rightarrow \quad \frac{3 }{2}a^2 = 1 \quad \rightarrow \quad a = \sqrt{\frac{2 }{3 }}  
\end{gather*}
Por lo tanto obtenemos que: 
\caja{red}{}{
  \begin{gather*}
    a = \sqrt{\frac{2 }{3 }} \qquad \qquad b = -\frac{1}{\sqrt{3 } } 
  \end{gather*}
}

% Punto 2 %%%%%%%%%%%%%%%%%%%%%%%%%%%%%%%%%%%%%%%%%%%%

\section{}

Una particula se encuentra en movimiento unidimensional en el eje $ x  $ y es representada en el estado $ \ket{\psi} $, con función de onda
\begin{gather*}
  \bra{x }\ket{\psi} = \psi(x) 
\end{gather*}
En el tramo $ s  $ del eje $ x  $, definido como 
\begin{gather*}
  a \leq x \leq b  
\end{gather*}
Se aproxima la función de onda con la siguiente expresión: 
\begin{gather*}
  \psi(x) =  A \cdot (1+ \frac{x }{d }) 
\end{gather*}
Tenemos los siguientes valores numéricos: 
\begin{gather*}
  a = 2 \AA, \quad b = 3 \AA, \quad d = 1 \AA
\end{gather*}
Determinar la constante real y positiva $ A  $ (valor numérico y unidades) de manera que la probabilidad de encontrar la partícula en el tramo $ s  $ sea igual a $ \frac{1}{3 } $.

\textbf{Solucion: }

Necesitamos que $ \bra{\psi}\ket{\psi} = \frac{1}{3}  $, introduciendo la identidad: 
\begin{gather*}
  \displaystyle\int_{a }^{b } \bra{\psi}\ket{x }\bra{x }\ket{\psi}dx = \frac{1}{3}\\
  \displaystyle\int_{a }^{b } \left|\psi(x) \right|^2 dx = \frac{1}{3 } = \displaystyle\int_{a }^{b } A^2 \left(1 + \frac{x }{d }\right)^2 dx = \frac{1}{3}
\end{gather*}
Resolviendo la integral: 
\begin{gather*}
  \left. A^2 d \left[\frac{\mu^3 }{3 }\right]\right|_a^b = \frac{1}{3}\\
  A^2d \left[\left(1 + \frac{d }{b }\right)^3 - \left(1 + \frac{a }{d }\right)^3 \right] = 1
\end{gather*}
Reemplazando $ a,b,d  $: 
\begin{gather*}
  A^2 \left(37 \right) = 1
\end{gather*}
Entonces: 
\caja{red}{}{
  \begin{gather*}
    A = \frac{1}{\sqrt{37 } }\AA ^ {\frac{1}{2}} 
  \end{gather*}
}

% Punto 3 %%%%%%%%%%%%%%%%%%%%%%%%%%%%%%%%%%

\section{}
Una particula se encuentra en movimiento en el espacio y es representada por el estado $ \ket{\psi} $. 

Su función de onda, que depende de $ r  $ y no de lo ángulos, es: 
\begin{gather*}
  \bra{\vec r }\ket{\psi} = \psi(\vec r ) = A \frac{1}{r} e^{- \frac{r }{2d }}  
\end{gather*}
Determinar la constante $ A  $, real y positiva, de manera que el estado $ \ket{\psi} $ sea normalizado a 1, es decir $ \bra{\psi}\ket{\psi} = 1  $.

\textbf{Solucion: }

Para normalizar el estado $ \ket{\psi} $, introducimos la identidad en $ \bra{\psi}\ket{\psi}  =1 $: 
\begin{align*}
  \displaystyle\int_{-\infty}^{\infty}\bra{\psi}\ket{\vec r } \bra{\vec r }\ket{\psi} d^3 x &= \displaystyle\int_{- \infty}^{\infty} \frac{A^2 }{r^2 } e ^ {- \frac{r }{d }} d^3x = \displaystyle\int_{}^{}d\Omega \displaystyle\int_{- \infty}^{\infty}\frac{A^2 }{r^2 } e ^ {- \frac{r }{d }} r^2 dr\\
  &= 4\pi A^2 \displaystyle\int_{0 }^{\infty} e ^ {- \frac{r }{d }}dr = 4\pi A^2 \left[-d e ^ {-\frac{r }{d }}\right]_0^\infty = 1\\
  4 \pi A^2 dA^2 = 1 
\end{align*}
Por lo tanto: 
\caja{red}{}{
  \begin{gather*}
    A = \frac{1}{\sqrt{4\pi d } } 
  \end{gather*}
}

% Punto 4 %%%%%%%%%%%%%%%%%%%%%%%%%%%%%%%%%%%%%%%%%%%%%

\section{}
Una partícula se encuentra en movimiento unidimensional, en un potencial de oscilador armónico. 

\hfill

El hamiltoniano tiene la forma estándar: 
\begin{gather*}
  H = \frac{p^2 }{2m } + \frac{1}{2}m \omega^2 x^2  
\end{gather*}
considerar el estado 
\begin{gather*}
  \ket{\psi} = \frac{1}{2}\ket{0 } + \frac{\sqrt{3 } }{2 }\ket{2 } 
\end{gather*}
Calcular, el estado $ \ket{\psi} $, los valores esperados de los siguientes operadores: $ x, p , x^2, p^2, H  $.


\textbf{Solucion: }

Para $ \mathbf{x } $: 

Podemos reescribir los operadores en funcion de $ a  $ y $ a ^ {\dagger } $: 
\begin{gather*}
  x = \sqrt{\frac{\hbar }{2m\omega}} (a + a ^ {\dagger })\\
  \bra{\psi} x \ket{\psi} = \sqrt{\frac{\hbar }{2m \omega}} \bra{\psi}a + a ^ {\dagger }\ket{\psi}\\
  \bra{\psi}x \ket{\psi} = \left(\frac{\hbar }{2m \omega}\right)^ {\frac{1}{2}} (\bra{\psi}a \ket{\psi} + \bra{\psi}a ^ {\dagger }\ket{\psi}  )
\end{gather*}
como $ a \ket{n }  = \sqrt{n } \ket{n-1 }$  y $ a ^ {\dagger }\ket{n } = \sqrt{n + 1 } \ket{n + 1 } $ entonces: 
\begin{gather*}
  \bra{n - 1 }a \ket{n } = \sqrt{n } \qquad \qquad 
  \bra{n + 1 }a ^ {\dagger }\ket{n } = \sqrt{n + 1 }  
\end{gather*}
Entonces: 
\begin{gather*}
  \bra{\psi}a \ket{\psi} =   
\end{gather*}

\end{document}
