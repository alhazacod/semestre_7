\documentclass{article}

\usepackage{physics}
\usepackage{graphicx}
\usepackage{float}
\usepackage{amsmath}
\usepackage{amssymb}


\usepackage[utf8]{inputenc}
\usepackage[a4paper, margin=1in]{geometry} % Controla los márgenes

\begin{document}

\textbf{Integrantes: }Manuel Garcia, Carlos Llanos, Angel Almonacid.

\hfill


\section{ejercicio 5.13 }
\[
\begin{aligned}
  & x = r\cos{\theta} \\ 
  & y = r\sin{\theta}
\end{aligned} \quad \quad \quad \begin{aligned} & r = \sqrt{ x^2 + y^2 } \\ & \theta = \arctan{ \left( \frac{y}{x} \right) } \end{aligned}
\]

\[
\begin{gathered}
  \begin{aligned}
    & \partial_r x = c_{\theta} \\ 
    & \partial_r y = s_{\theta} 
  \end{aligned} \quad \quad \begin{aligned}
    & \partial_{\theta} x = -rs_{\theta} \\ 
    & \partial_{\theta} y = rc_{\theta}
\end{aligned} \quad \quad \begin{aligned}
    \partial_x r = \frac{x}{r} \quad \quad \partial_y r = \frac{y}{r} \\
    \begin{aligned}
      & \partial_x \theta = \frac{ \frac{-y}{x^2} }{ 1 + \frac{y^2}{x^2} } = \frac{-y}{ x^2 + y^2 } \\
      & \partial_y \theta = \frac{x}{ x^2 + y^2 }
\end{aligned}
\end{aligned} \\
\begin{aligned}
  dx \wedge dy = (c_{\theta} dr + - rs_{\theta} d\theta ) \wedge (s_{\theta}dr + rc_{\theta} d\theta ) = rc_{\theta}^2 dr \wedge d\theta + rs_{\theta}^2 dr \wedge d\theta = rdr \wedge d\theta
\end{aligned}
\end{gathered}
\]


\section{Ejercicio 5.15}
\begin{gather*}
  \xi \in \Omega ^ {q }(M) \qquad \qquad \omega\in \Omega ^ {r }\\
  \xi \land \omega = \xi _{i_1, \cdots, i_q }\omega _{j_1, \cdots, j_r } dx ^ {i_1 } \land \cdots \land dx ^ {i_q }\land dx ^ {j_1}\land \cdots \land dx ^ {j_r }\\
  d(\xi \land \omega) = \frac{1}{(q+r)!} \frac{\partial (\xi _{i_1, \cdots, i_q } \omega _{j_i,\cdots,J_r )} }{x ^ {v }} dx ^ {v }\land dx ^ {i _1}\land \cdots \land dx ^ {i_q }\land dx ^ {j_1 }\land \cdots \land dx ^ {j_r }\\
  = \frac{1}{q! } \frac{\partial \xi _{i_1, \cdots, i_q }  }{\partial x ^ {v }}dx ^ {v } \land dx ^ {i_1 }\land \cdots \land dx ^ {i_q }\land \omega_{j_1 , \cdots j_r  } dx ^ {j_1 }\land \cdots \land dx ^ {i_r }
\end{gather*}

\begin{gather*}
  \xi = f dx_{i_1}\land \cdots \land dx _{i_q } = f dx _{I }  \qquad \qquad \omega = g dx _{J } 
\end{gather*}

\begin{gather*}
  d(\xi \land \omega ) = d(f dx_I \land g dx_J ) = d\xi \land \omega + \xi \land d\omega 
\end{gather*}

\begin{gather*}
  X[\omega(Y)] - Y[\omega(X)] - \omega([X,Y]) = \frac{\partial \omega_\mu  }{\partial x^v } (X^vY^\mu- X^\mu Y^v )\\
\end{gather*}

\begin{gather*}
  \omega([X,Y]) = \omega_v(X^\mu \partial_\mu Y^v - Y^\mu \partial_\mu X^v )\\
  X[\omega(Y)] = X^\mu \partial_\mu \omega_v Y^v + Y^\mu \omega_v \partial_\mu X^v\\
  Y[\omega(X)] = Y^\mu \partial_\mu \omega_v X^v + Y^\mu \omega_v \partial_\mu X^v\\
\end{gather*}

\begin{align*}
  X[\omega(Y)] - Y[\omega(X)] - \omega([X,Y]) &= X^\mu \partial_\mu \omega_v Y^v - Y^\mu \partial_\mu \omega_v X^v = \partial_\mu \omega_v (X^\mu Y^v - Y^\mu X^v )\\
  &= \partial_\mu \omega_v (X^\mu Y^v - Y^\mu X^v )
\end{align*}

\begin{gather*}
  d\omega (X_1, \cdots X _{p+1 } ) = \displaystyle\sum_{i=1 }^{r }(-1) ^ {i + 1 } X_i \omega (X_1, \cdots \hat X_i \cdots X _{i+1 } ) + \displaystyle\sum_{i<j }^{} (-1)^ {i + 1 } \omega ([x_i, X_j], X_1, \cdots \hat X_i, \cdots \hat X_j \cdots X _{i + 1} ) 
\end{gather*}

r-forma $ \omega \in \Omega^r(M) $ 
\begin{gather*}
  \omega = \frac{1}{r! } \omega_{\mu_1 \cdots \mu_r } dx ^ {\mu_1 }\land \cdots \land dx ^ {\mu_r }\\
  d\omega = \frac{1}{r! }\left(\frac{\partial  }{\partial x^v }\omega _{\mu_1 \cdots \mu_r } \right) dx ^ {v }\land dx ^ {\mu_1 } \land \cdots \land dx ^ {\mu_r }
\end{gather*}




\section{Ejercicio 5.19 }
\begin{enumerate}
\item[(a)] $\mathbb{R}^+ = \lbrace x \in \mathbb{R} | x > 0 \rbrace$ \quad \quad \, $\frac{ \partial }{ \partial x} x^{-1} = -x^{-2} \neq 0 $ \, 

\item[(b)] 
\[
\begin{gathered}
\partial_x z  = \partial_x ( x+y)  = 1   \\
\partial_x( x^{-1} ) = \partial_x (-x) = -1 
\end{gathered}
\] \quad 
\item[(c)] \[
\begin{gathered}
  (a,b) + (x,y) = (a+x, b+y) \quad \quad \, Dg =Dg(x) = \left[ \begin{matrix} a & \\ & b \end{matrix} \right] \\
  (x,y)^{-1} = (-x, -y) \quad \quad \, D(x,y)^{-1} = \left[ \begin{matrix} -1 & \\ & -1 \end{matrix} \right]
\end{gathered}
\]
\end{enumerate}

Grupo de Lorentz
\[
O(1,3) = \lbrace M \in GL(4,\mathbb{R}) | M \eta M^T = \eta \rbrace  \quad \quad \eta = \text{diag}{ ( -1, 1, 1, 1 ) }
\]
\end{document}
