\documentclass{article}

\usepackage[most]{tcolorbox}
\usepackage{physics}
\usepackage{graphicx}
\usepackage{float}
\usepackage{amsmath}
\usepackage{amssymb}


\usepackage[utf8]{inputenc}
\usepackage[a4paper, margin=1in]{geometry} % Controla los márgenes
\usepackage{titling}

\title{Clase 26 }
\author{Manuel Garcia.}
\date{\today}

\renewcommand{\maketitlehooka}{%
  \centering
  \vspace*{0.05cm} % Espacio vertical antes del título
}

\renewcommand{\maketitlehookd}{%
  \vspace*{2cm} % Espacio vertical después de la fecha
}

\newcommand{\caja}[3]{%
  \begin{tcolorbox}[colback=#1!5!white,colframe=#1!25!black,title=#2]
    #3
  \end{tcolorbox}%
}

\begin{document}
\maketitle

\section{Haces fibrados }
Con la variedad asociamos: $(\underset{Variedad total }{E }, \pi, M, \underset{Estructura (fibra) }{F },\underset{Grupo bajo la fibra }{G })$. Tambien lo representamos como $ E \overset{\pi }{\rightarrow }M  $
\section{Haz Vectorial }
$(\underset{Variedad total }{E }, \pi, M, \underset{V }{F },\underset{t _{ij } }{G })$. $(E , \pi, M, \mathbb{R}^ {k },GL(k, \mathbb{R}))$.
\subsection{Haz normal (a una variedad)}
Tambien es un ejemplo de haz vectorial. Está definido por las estructuras: Primero tomamos una variedad $ M  $ que vamos a encajar dentro de (embedded) en un espacio $ \mathbb{R}^ {m+k } $ y dentro de este vamos a definir un producto escalar $ \bra{V_1 }\ket{V_2 } = \delta _{ij } V _{1 } ^ {i } V _{2 } ^ {j }  $. Vamos a definir la variedad normal $ N _{p } M = \text{ Espacio normal a }T _{p } M \qquad N_p M \rightarrow u \in \mathbb{R}^ {m+k } / \bra{U}\ket{V } = 0 $. Haz fibrado: 
\begin{gather*}
  NM = \underset{p }{\cup }N _{p } M  
\end{gather*}
\subsection{Haz cotangente }
\begin{gather*}
  T* M = \underset{p }{\cap }T _{p } * M  
\end{gather*}
Haces: $ \{dx ^ {\mu  }\} $, $ \mu = 1,...,m  $
\end{document}
