\documentclass{article}

\usepackage[most]{tcolorbox}
\usepackage{physics}
\usepackage{graphicx}
\usepackage{float}
\usepackage{amsmath}
\usepackage{amssymb}


\usepackage[utf8]{inputenc}
\usepackage[a4paper, margin=1in]{geometry} % Controla los márgenes
\usepackage{titling}

\title{Clase 13}
\author{Manuel Garcia.}
\date{\today}

\renewcommand{\maketitlehooka}{%
  \centering
  \vspace*{0.05cm} % Espacio vertical antes del título
}

\renewcommand{\maketitlehookd}{%
  \vspace*{2cm} % Espacio vertical después de la fecha
}

\newcommand{\caja}[3]{%
  \begin{tcolorbox}[colback=#1!5!white,colframe=#1!25!black,title=#2]
    #3
  \end{tcolorbox}%
}

\begin{document}
\maketitle

\section{Vectores en variedades}
\begin{gather*}
  X = X ^ {i } \frac{\partial  }{\partial x ^ {i }} \qquad \qquad X(f) = \left. X ^ {i } \frac{\partial  }{\partial x ^ {i }}f \right|
\end{gather*}
$ X ^ {i } $ son las componentes de $ X  $, $ \{\frac{\partial  }{\partial x ^ {i }}\}  = \{e_i \} \quad \rightarrow $ Bases de $ T_p M  $.

$ X \quad \rightarrow \quad  $operador en $ T_p M  $. 
\begin{gather*}
  (r,\theta ) = (1, \frac{\pi}{4 })\qquad \qquad (x,y) = \left(\frac{\sqrt{2 } }{2}, \frac{\sqrt{2 } }{2 }\right) 
\end{gather*}
Nosotros podemos expresar unas coordenadas en terminos de las otras: 
\begin{gather*}
  V = V ^ {i } \frac{\partial  }{\partial x ^ {i }} = \bar V ^ {i } \frac{\partial  }{\partial y ^ {i }}  
   = \hat V ^ {i } \frac{\partial x ^ {k } }{\partial y ^ {i }} \frac{\partial  }{\partial x ^ {k }} \quad \rightarrow \quad V ^ {k } = \hat V ^ {i } \frac{\partial x ^ {k } }{\partial y ^ {i }}
\end{gather*}

\section{Covectores, vectores duales o 1-formas y espacio cotangente $ T_p^*M  $}
\textbf{1-forma } $ \omega: T_pM \quad \rightarrow \quad \mathbb{R} $, vamos a definirlo ahora como un producto escalar: 
\begin{gather*}
  \bra{\omega}\ket{V }: T_p M \quad \rightarrow \quad \mathbb{R}  
\end{gather*}
El diferencial de una función actual linealmente en $ T_p^*  $. El diferencial d euna función $ f  $ es una 1-forma definida en la variedad: $ \bra{df }\ket{V } \equiv V[f] = V ^ {i } \frac{\partial f  }{\partial x ^ {i }}\in \mathbb{R} $.

Vamos a definir el diferencial en coordenadas $ x  $: $ df = \frac{\partial f  }{\partial x ^ {i }}dx ^ {i } = f_i dx ^ {i }$. Vamos a llamar $ f_i  $ las componentes de $ df  $ y a $ \{dx ^ {i }\} $ las bases de $ T_p ^ {* }M  $.

Utilizando las definiciones anteriores obtenermos la siguiente identidad: 
\begin{gather*}
  \bra{dx ^ {j }}\ket{\frac{\partial  }{\partial x ^ {i }}} = \frac{\partial x ^ {j } }{\partial x ^ {i }} = \delta _{i } ^ {j }
\end{gather*}
ya que $ V ^ {j } \frac{\partial f  }{\partial x ^ {i }} \underset{ = \delta _{j } ^ {i }}{\bra{d x ^ {i }}\ket{\frac{\partial  }{\partial x ^ {i }}}} = V ^ {k } \frac{d  }{d x ^ {k }}f   $.

En resumen las 1-formas son objetos que tienen el indice abajo y que al operarse con los objetos de la variedad obtenemos un escalar.

\textbf{1-forma arbitraria: }
\begin{gather*}
  \omega = \omega_i dx ^ {i }
\end{gather*}
$ \omega_i  $ son las componentes de $ \omega  $ en la base $ dx ^ {i } $.
Producto interno: 
\begin{gather*}
  \bra{\omega }\ket{V } = \omega_i V ^ {j } \bra{dx ^ {i }}\ket{\frac{\partial  }{\partial x ^ {j }}} = \omega_i V ^ {i }: \quad T_p ^ {* }M \otimes T_p M \quad \rightarrow \quad \mathbb{R}
\end{gather*}
\begin{gather*}
  \bra{\omega}\ket{V } = \bra{\omega_i dx ^ {i }}\ket{V ^ {k } \frac{\partial  }{\partial x ^ {k }}}\\
  = \omega_i V ^ {k }\bra{dx ^ {i }}\ket{\frac{\partial  }{\partial x ^ {k }}} = \omega_i V ^ {i }\\
  \bra{\omega}\ket{V } = \omega_i V^i  
\end{gather*}
Y tenemos que: 
\begin{gather*}
  \bra{\omega_1 + \omega_2 }\ket{V } = \bra{\omega_1 }\ket{V }+ \bra{\omega_0 }\ket{V }\\
  \bra{\omega }\ket{V_1V_2 } = \bra{\omega}\ket{V_1 } + \bra{\omega}\ket{V_2 }   
\end{gather*}

\hfill

\hfill 

\hfill 

\begin{gather*}
  \omega = \omega_i dx ^ {i } = \hat \omega_i dy ^ {i }
  = \hat \omega_i \frac{\partial y ^ {i } }{\partial x ^ {k }}dx ^ {k } = \hat \omega_j \frac{\partial y ^ {i } }{\partial x ^ {i }} dx ^ {i }\\ 
  \omega_i = \hat \omega_j \frac{\partial y ^ {j } }{\partial x ^ {i }} \quad \rightarrow \quad \hat \omega_j = \omega_i \frac{\partial x ^ {i } }{\partial y ^ {j }}
\end{gather*}

\section{Tensores en los productos de los espacios tangente sy cotangentes }
Tensor tipo $ (q,r ) $
\begin{gather*}
  T = T ^ {\mu_1, \mu_2...\mu_q} _{v_1,v_2,...v_e } \frac{\partial  }{\partial x ^ {\mu_1 }} \otimes ... \otimes \frac{\partial  }{\partial x ^ {\mu_q }}\otimes dx ^ {\nu_1 } \otimes ... \otimes dx ^ {\nu _r }
\end{gather*}
Funcion multilineal: $ \otimes ^ {q } T_p M \otimes ^ {r }T _p ^ {* }M \quad \rightarrow \quad \mathbb{R} $.
Accion de un tensor sobre los vectores $ V_a  $ y las formas $ \omega_b  $:
\begin{gather*}
  T(\omega_2, \cdots, \omega_r; V_1, \cdots, V_q) = T ^ {\mu_1,...\mu_r} _{\nu_1,...\nu_q } \omega_{1}_{\mu_1} \cdots \omega _{r}_{\mu_r}  V_1 ^ {\nu_1,...,\nu_q }_{q } \cdots V_p ^ {\nu_r}
\end{gather*}
\subsubsection{Comportamiento de tensores ante mapeos }
Sea $ f  $ un mapa entre $ M  $ y $ N  $. Vamos a comar los espacios tangentes y un punto en este espacio tangente en cada variedad y tenemos una funcion $ f  $ que va de uno a otro. $ T_p M \rightarrow T _{f(p)} N  \underset{g }{\rightarrow  }\mathbb{R}$. 
\begin{gather*}
  f: M \rightarrow N \rightarrow f_* : T_p M \rightarrow T _{f (p) } N  
\end{gather*}
$ f_*  $ es el mapeo diferencial.

\hfill 

\hfill 

$ V \in T_p M \rightarrow f_* V \in T _{f(p)} N  $.
\begin{gather*}
  (f_* V )[g] \equiv V[g \circ f]\qquad \qquad g\circ f \in \mathcal F (M) 
\end{gather*}

\end{document}
