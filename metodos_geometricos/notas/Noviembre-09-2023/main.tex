\documentclass{article}

\usepackage[most]{tcolorbox}
\usepackage{physics}
\usepackage{graphicx}
\usepackage{float}
\usepackage{amsmath}
\usepackage{amssymb}


\usepackage[utf8]{inputenc}
\usepackage[a4paper, margin=1in]{geometry} % Controla los márgenes
\usepackage{titling}

\title{Clase 22 }
\author{Manuel Garcia.}
\date{\today}

\renewcommand{\maketitlehooka}{%
  \centering
  \vspace*{0.05cm} % Espacio vertical antes del título
}

\renewcommand{\maketitlehookd}{%
  \vspace*{2cm} % Espacio vertical después de la fecha
}

\newcommand{\caja}[3]{%
  \begin{tcolorbox}[colback=#1!5!white,colframe=#1!25!black,title=#2]
    #3
  \end{tcolorbox}%
}

\begin{document}
\maketitle

\section{Conexión de levi-civita }
Compatible con $ g: \quad \grad _{\alpha } g _{\mu\nu} = 0 \qquad \qquad \Gamma _{\mu\nu} ^ {\alpha} = \Gamma _{\nu\mu} ^ {\alpha} \rightarrow \text{Simetrica } $.

\begin{gather*}
  \bar \Gamma _{\mu\nu} ^ {\alpha} = \begin{bmatrix}
      \alpha \\
      \mu\nu 
  \end{bmatrix} + K _{\mu\nu} ^ {\alpha}\\
  \Delta \Gamma _{\mu\nu} ^ {\alpha} = t _{\mu\nu} ^ {\alpha} \quad \rightarrow \quad \Gamma _{\mu\nu} ^ {\alpha} = \hat \Gamma _{\mu\nu} ^ {\alpha} + t _{\mu\nu} ^ {\alpha}\\
  t _{\mu\nu} ^ {\alpha} = -K ^ {\alpha} _{\mu\nu}  \quad \rightarrow \quad 
  \Gamma _{\mu\nu} ^ {\alpha} = \begin{bmatrix}
      \alpha \\
      \mu\nu 
  \end{bmatrix} 
\end{gather*}

\section{Componentes independientes del tensor de Riemann }
Podemos determinar la cantidad de componentes independientes del tensor de Riemann en una variedad $ M  $, $ dim(M )= m $, teniendo en cuenta sus simetrías: 
\begin{align*}
  \text{(1)}\quad R _{\sigma\rho\mu\nu} = - R _{\sigma\rho\nu\mu}  \qquad \qquad  &\text{(2)} \quad R _{\sigma\rho\mu\nu} = - R _{\rho\rho\mu\nu} \\
\text{(3)}\quad R _{\sigma\rho\mu\nu} =  R _{\mu\nu\sigma\rho}  \qquad \qquad  &\underset{Identidad de Bianchi }{\text{(4)} \quad R _{\alpha \mu\nu\sigma}  + R _{\alpha\nu\sigma\mu} + R _{\alpha\sigma\mu\nu} = 0 }
\end{align*}
Solo vale para la conexion de levi-civita.

\hfill 
\hfill 

\hfill 

\begin{gather*}
  R _{\mu\alpha\nu} ^ {\alpha} = R _{\mu\nu}  \quad \rightarrow \quad G _{\mu\nu} = \frac{8 \pi G }{c ^ {4 }}T _{\mu\nu}  
\end{gather*}

\subsection{Ecuaciones de campo relatividad general }
\begin{gather*}
  \grad ^\mu R _{\mu\nu} = \kappa \grad ^ {\mu} T _{\mu\nu}  =0 \quad \rightarrow \quad  \\
  R _{\mu\nu} - \frac{1}{2}g _{\mu\nu} R = \kappa T _{\mu\nu} \quad \rightarrow  \quad \text{Ec. de campo Einstein} 
\end{gather*}

\begin{gather*}
  \mathbb{L} = \sqrt{-g } R, \quad \rightarrow \quad S _{EH } = \displaystyle\int_{}^{} \sqrt{-g }R d ^ {4 }x \quad \rightarrow \quad \text{Accion de Einstein-Hilbert}   
\end{gather*}

¿Por qué este lagrangiano? Hasta en la teoria mas basica tenemos $ g _{\mu\nu} , \quad \Gamma _{\mu\nu} ^ {\alpha} = g(\partial g ) $. Podemos obtener $ R _{\mu\nu\sigma} ^ {\alpha} $. Con los objetos $ R _{\alpha\mu\nu\sigma} , \quad g _{\mu\nu}  $ debemos construir un escalar. Escalar de ricci $ R _{\nu\beta} g ^ {\nu\beta} = R  $.

\end{document}
