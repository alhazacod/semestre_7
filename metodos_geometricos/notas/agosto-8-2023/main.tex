\documentclass{article}

\usepackage[most]{tcolorbox}
\usepackage{physics}
\usepackage{graphicx}
\usepackage{amsmath}
\usepackage{amssymb}

\usepackage[utf8]{inputenc}
\usepackage[a4paper, margin=1in]{geometry} % Controla los márgenes
\usepackage{titling}

\title{Clase 1 }
\author{Manuel Garcia.}
\date{\today}

\renewcommand{\maketitlehooka}{%
  \centering
  \vspace*{0.05cm} % Espacio vertical antes del título
}

\renewcommand{\maketitlehookd}{%
  \vspace*{2cm} % Espacio vertical después de la fecha
}

\newcommand{\caja}[3]{%
  \begin{tcolorbox}[colback=#1!5!white,colframe=#1!25!black,title=#2]
    #3
  \end{tcolorbox}%
}

\begin{document}
\maketitle

\section{Coordenadas}
Geometria euclideana $ \rightarrow $ geom. analitica o de coordenadas $ \rightarrow  $ geom. diferencial.

\section{espacio cartesiano }
$ p \rightarrow (x^i_p)  $, $ i=1,...,n $
\subsection{transformacion de coordenadas. } % (fold)
\begin{gather}
P_0 = \begin{bmatrix} x_0^1  & ...  & x_0^n  \end{bmatrix} \qquad \text{conjunto abierto D}\\
  x_0^1-x^i<\in \rightarrow  \text{Abierto }  
  \label{eq:transf-coordenadas }
\end{gather}

En una region D:
\begin{gather}
  (x^i) = \begin{bmatrix} x^1  & ...  & x^n \end{bmatrix} \text{Donde }i = 1,...,n 
\end{gather}
En la misma region: 
\begin{gather}
  (z^i)= \begin{bmatrix} z^1  & ...  & z^n \end{bmatrix}
\end{gather}
Transformacion suave:
\begin{gather}
  x^i = x^i\begin{bmatrix} z^1  & ...  & z^n \end{bmatrix}
\end{gather}
\caja{green}{Def. }{
  un punto $ P_0 = \begin{bmatrix} x_0^1  & ...  & x_0^n  \end{bmatrix}  $ es ordinario o no singular de z si:
  \begin{gather}
a_j^i = ( \frac{\partial x^i  }{\partial z_i} ) _{z^i=z_0^i} 
  \end{gather}
  es no singular. 
}
\subsection{transformada lineal:}
$ x^i = a_j^i z^i $ Notacion de einstein.
\caja{green}{Teorema de la funcion inversa }{
  las coordenadas z se pueden expresar en teminos de x si la matrix $ a_j^i  $ es invertible. es decir si existe $ b_j^i  $ tal que: 
  \begin{gather}
     a_j^i \dot b_k^j = \delta_k^i
  \end{gather}
}
 ej: esfericas
 \begin{gather}
    z^1 = r \qquad z^2 = \theta \qquad z^3 = \phi
 \end{gather}
 Transformada: $ x^1 = r \sin{\theta }\cos{\theta} \qquad x^2 = r \sin{\theta}\cos{\phi} \qquad x^3=r \cos{\phi}$
 \begin{gather}
   J =  det(J_j^i) = r^2 \sin{\theta} \neq 0 \qquad para \qquad r \neq 0,\theta \neq 0, \phi \neq \pi    
 \end{gather}

% subsection transformacion de coordenadas.  (end)

% section espacio cartesiano $ p \rightarrow (x^i_p)  $, $ i=1,...,n $ (end)
 \section{Espacio euclideano} % (fold)
 \label{sec:Espacio euclideano}
 Si la distancia entre dos puntos está dada por:
 \begin{gather}
  d^2 = (x^1-y^1)^2+ (x^2+y^2)^2 + (x^3+y^3)^2 
  \label{eq:distancia_euclideana}
 \end{gather}
 entonces es un espacio euclideano 
 \caja{red}{Vectores y bases en $ \mathbb{R}^3 $}{
  \begin{gather}
    \{ e_1\}= e_1,e_2,e_3 \rightarrow (1,0,0),(0,1,0),(0,0,1) \\
    v = x^1 e_1 + x^2 e_2 + x^3 e_3 = x^i e_i
    \label{eq:bases_r3}
  \end{gather}
 }
 \caja{red}{Producto escalar}{
  Sea $ v = (x^1,...,x^n) $, $ w = (y^1,...,y^n ) $
  \begin{gather}
    \bra{v}\ket{\omega} \equiv \displaystyle\sum_{i=1 }^{n }x^i y^j  
    \label{eq:producto_escalar}
  \end{gather}
  propiedades: 
  \begin{itemize}
    \item conmutativo $ \bra{v}\ket{\omega}=\bra{\omega}\ket{v}   $
    \item distributivo $ \bra{v}\ket{a\omega + b u } = a \bra{v}\ket{\omega} + b \bra{v}\ket{u }  $ 
    \item $ \bra{v}\ket{v}>0 \qquad sii \qquad v \neq 0    $
  \end{itemize}
  \tcblower
  \begin{gather}
    |v|^2 = \bra{v}\ket{v} \qquad \qquad \cos{\alpha} = \frac{\bra{v}\ket{\omega} }{\sqrt{\bra{v}\ket{v} \bra{\omega}\ket{\omega}  }  }   
    \label{eq:v2_y_angulo}
  \end{gather}
 }
 \caja{black}{curvas}{
  Una curva parametrica diferenciable en $ \mathbb{R}  $ es una funcion diferenciable en un intervalo $ I = (a,b) $ en $ \mathbb{R}^n, a(t) : I \rightarrow \mathbb{R}^n    $
  \begin{gather}
     a(t) = \begin{bmatrix} f^1(t) & ...  & f^n(t) \end{bmatrix} 
  \end{gather}
  Una curva es regular si $ v(t)\neq 0   $ para todo $ t\in I $
 }
 \caja{red}{longitud de arco }{
  \begin{gather}
    S = \int_{a }^{b }\sqrt{\bra{v(t)}\ket{v(t)}dt  }
    \label{eq:long_arco_euclideo}
  \end{gather}
 }
 % section Espacio euclideano (end)

\end{document}
