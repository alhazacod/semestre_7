\documentclass{article}

\usepackage[most]{tcolorbox}
\usepackage{physics}
\usepackage{graphicx}
\usepackage{float}
\usepackage{amsmath}
\usepackage{amssymb}


\usepackage[utf8]{inputenc}
\usepackage[a4paper, margin=1in]{geometry} % Controla los márgenes
\usepackage{titling}

\title{Clase 11}
\author{Manuel Garcia.}
\date{\today}

\renewcommand{\maketitlehooka}{%
  \centering
  \vspace*{0.05cm} % Espacio vertical antes del título
}

\renewcommand{\maketitlehookd}{%
  \vspace*{2cm} % Espacio vertical después de la fecha
}

\newcommand{\caja}[3]{%
  \begin{tcolorbox}[colback=#1!5!white,colframe=#1!25!black,title=#2]
    #3
  \end{tcolorbox}%
}

\begin{document}
\maketitle

\section{Espacios Lineales}
\subsection{Producto interno }
Es una regla de asociacion que tenemos entre los vectores $ g: V \rightarrow V ^ {* } $ y $ g _{ij } \in GL(n, \mathbb{R}) $ ($ GL  $ es el grupo de tranformaciones lineales). 
\begin{gather*}
  v ^ {i }\rightarrow g _{ij } v ^ {j } = v _{i } ^ {* }= v _{i} 
\end{gather*}
Podemos introducir la notacion e Einstein. 
\caja{red}{}{
  \begin{gather*}
    g(\vec v_1,\vec v_2) = \bra{g v_1 }\ket{v_2 } = g _{ij } v_1 ^ {i } v_2 ^ {j }\\
    \text{Esta es la definicion de producto interno.}
  \end{gather*}
  Toma dos vectores y los convierte en un numero. 
  \begin{gather*}
    g(v_1,v_2): V. V \rightarrow \mathbb{R} 
  \end{gather*}
}
Si $ g _{ij }  = g _{ji }  $ y ademas $ g(v_1,v_2) \geq 0  $ (esto es una estructura metrica). 

\subsubsection{Adjunto}
\begin{gather*}
  W(n, \mathbb{R}),\quad \{f_a \} , \quad\underset{\text{Isomorfismo}}{G}: W \rightarrow V\\
  f: V \rightarrow W \quad \text{adjunto del mapeo }f = \hat f\\
  G(W,fV) = g (v, \hat f W)\\
  G(W,fV) = W ^ {\alpha} G _{\alpha\beta} f ^ {\beta}_i v ^ {i } = g _{ij } v ^ {j } \hat f_\beta^i w ^ {\beta }\\
  w ^ {\alpha} G _{\alpha\beta} f _{i } ^ {\beta} v ^ {i } = g _{ij } v ^ {i }\hat f ^ {j }_{\alpha} w ^ {\alpha}
\end{gather*}
\caja{red}{}{
  \begin{gather*}
    G _{\alpha\beta} f _{i } ^ {\beta} = g _{ij } \hat f _{\alpha} ^ {j } 
  \end{gather*}
}
Ahora veremos que $ G _{\alpha\beta} = \delta _{\alpha\beta } \qquad g _{ij } = \delta _{ij }  $, entonces: 
\caja{red}{}{
  \begin{gather*}
    f _{i } ^ {\alpha} = \hat f _\alpha ^ {i }\\
    \hat f = f ^ {T }
  \end{gather*}
  Condiciones: $ \quad G _{\alpha\beta} = \delta _{\alpha\beta } \quad g _{ij } = \delta _{ij } $
}

\subsection{Tensores }
\begin{gather*}
  f: V \rightarrow \mathbb{R}\qquad f \text{ es lineal.} \rightarrow f\in V ^ {* }\\
  g: V \otimes V \rightarrow \mathbb{R} \qquad g \text{ es lineal.}\\
  T: V \underset{p-copias}{\otimes ... \otimes} V \otimes V^*\underset{q-copias}{\otimes ... \otimes} V^* \rightarrow \mathbb{R}\\
  \tau(\xi_1, ..., \xi_p, \eta_1, ..., \eta_q ) = \# \in \mathbb{R}. \quad \text{Con }\xi _i \in V \text{ y } \eta_i \in V^*\\
  \tau(\alpha \xi_1 + \beta \xi_1', [\xi],[\eta]) = \alpha \tau(\xi_1 , [\xi],[\eta]) + \beta \tau(\xi_1', [\xi],[\eta])
\end{gather*}

\subsubsection{Operaciones entre tensores }
\begin{itemize}
  \item Suma: 
    \begin{gather*}
      \tau \in \mathcal{T}_q^p \\
      S\in \mathcal{T} _q^p\\
      \tau + S \in \mathcal{T}_q^p \\
      (\tau + S)(\xi_1,...,\xi_p,\eta_1,...\eta_q) = \tau(\xi_1,...,\xi_p,\eta_1,...\eta_q) + S(\xi_1,...,\xi_p,\eta_1,...\eta_q)
    \end{gather*}
  \item Producto tensorial
    \begin{gather*}
      \tau = M \otimes V, \quad M \in J ^p_q , \quad V\in J ^ {p' }_{q' } \\
      M(\omega_1,...,\omega_p ; u_1,...,u_q), \quad V(\xi_1,...,\xi _{p' } ; v_1,...,v _{q'} )\\
      \tau(\omega_1,...,\omega_p,\xi_1,...,\xi _{p' } ; u_1,...,u_q,v_1,...,v _{q'} ) \in \mathcal{T} _{q+q' } ^ {p+ p' }
    \end{gather*}
\end{itemize}

\section{Espacios topológicos }
Sea $ X  $ y sea $ \mathcal T = \{u_i|i\in I \} $ una colección de subconjuntos de $ X  $. Para llamarlo espacio topologico necesta cumplir 3 condiciones: 
\begin{itemize}
  \item $ \{\emptyset, X\} \in \mathcal T $
  \item $ \mathcal T _J = \{u_j|j \in J \} \qquad u_j \in \mathcal T _I   $ entonces tenemos que $ \underset{j \in J }{\cup } u_j \in \mathcal T_I $
  \item $ J_k = \{u_k | k \in k \} \qquad \underset{k\in K }{\cap } u_k \in \mathcal T_I $
\end{itemize}
\begin{gather*}
  (X,\mathcal T ) \rightarrow \text{Espacio topológico } 
\end{gather*}

\subsection{Métrica }
\begin{itemize}
  \item $ d(x,y) = d(y,x) $
  \item $ d(x,y) \geq 0  $ Solamente $ d(x,y) = 0  $ si $ x=y $
  \item $ d(x,y) + d(y,z) \geq d(x,z) $
\end{itemize}
Cuando cumpla la desigualdad triangular (2 y 3) es un espacio\textbf{ rimaniano}.

\hfill

\hfill

Utilizando 2 y 3 podemos obtener que para el caso \textbf{pseudo-rimaniano}: 
\begin{itemize}
  \item $ d(x,v) = 0 \qquad \forall v \in X \rightarrow x = 0  $
\end{itemize}

\textbf{Espacio de Hausdorff } es un espacio de Hausdorff si la union de dos entornos es vacio. $ U_x \cap U _{x' } = \emptyset $

\end{document}
