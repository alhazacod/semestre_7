\documentclass{article}

\usepackage[most]{tcolorbox}
\usepackage{physics}
\usepackage{graphicx}
\usepackage{amsmath}
\usepackage{amssymb}


\usepackage[utf8]{inputenc}
\usepackage[a4paper, margin=1in]{geometry} % Controla los márgenes
\usepackage{titling}

\title{Clase 2 }
\author{Manuel Garcia.}
\date{\today}

\renewcommand{\maketitlehooka}{%
  \centering
  \vspace*{0.05cm} % Espacio vertical antes del título
}

\renewcommand{\maketitlehookd}{%
  \vspace*{2cm} % Espacio vertical después de la fecha
}

\newcommand{\caja}[3]{%
  \begin{tcolorbox}[colback=#1!5!white,colframe=#1!25!black,title=#2]
    #3
  \end{tcolorbox}%
}

\begin{document}
\maketitle

\section{Observatorio Astronomico Nacional (OAN)}
¿Por qué no hay carrera de astronomia? Se hizo plan y todo pero no habia suficientes profesores ya que solo hay 10 profesores, ademas de que el pensum seria muy parecido al de la carrera de fisica, se decidió que no vale la pena. El camino deberia ser estudiar fisica para dedicarse a la astronomia. Solo existe el pregrado de astornomia en la universidad de antioquia.

El observatorio tiene una cupula grande, una cúpula pequeña y un cuarto oscuto.

\section{Grupos de investigacion}
\begin{itemize}
  \item Astornomia, astrofisica y cosmologia (Juan Manuel Tejeiro). 
  \item Astronomía gláctica, gravitación y cosmología (Leonardo Castañeda). 
  \item Analisis numérico y simulación en paralelo (Mario Armando Higuera Garzón). 
\end{itemize}
Las investigaciones de todo los grupos se dirigen hacia el de astronomia, astorfisica y cosmologia.

\section{Astrofísica Estelar }
Se encarga de estudiar la formacion de estrellas y como los campos magneticos influyen en la creacion de las estrellas. Como logra la presion, temperatura, etc. adecuada para convertirse en estrella. Y el como el material residual se comporta al rededor de las estrellas, como discos su dinamica. Es liderado por Giovanni Pinzón Estrada.

El objetivo es el estudio de la evolución del momento anular durante la formación de las estrellas, conexión con la generación del campo magnetico y el posible impacto sobre la formación planetaria. 

\hfill

\textbf{Proyectos en desarrollo }
\begin{itemize}
  \item spectroscopic monitoring of young stars with circumstellar disks, to investigate the impact of stellar activity on planet formation and evolution. 
  \item Estudios de estrellas jóvenes y sus discos protoplanetarios en la era GAIA - APOGEE - WISE.
\end{itemize}

En las actividades de extensión tambien existe el astroturismo.

\section{Astrofísica solar }
Se estudia la dinamica solar y todos los fenomes que ocurren en las estrellas. Los fenomenos involucrados son extremos en energia y tamaño.

Se estudia la fenomenología del Sol y su efecto en el clima espacial que ha despertado un creciente interés por las investigaciones en física solar. Grupo de Astrofisica Solar (GoSA). 

\section{Núcleos activos de galaxias }
Dentro de estas galaxias existen agujeros negros y son los objetos mas lejanos a pesar de ser los mas luminosos del cielo.

Profesores: Mario armando higuera, josé gregorio portilla 

Estudio de los procesos asociados a la emisión proveniente de las regiones centrales de las denominadas galaxias activas. Estos procesos involucran ampliar el conocimiento sobre la forma e intensidad de las emisiones y su interaccion con las estructuras internas del núcleo activo, así como la actividad estelar circumnuclear. 

\begin{itemize}
  \item Observaciones espectroscópicas en AGNs y su uso como indicadores de posibles escenarios de evolución. 
  \item AGNs 
\end{itemize}

\section{Astornomía galáctica, gravitación y cosmología }
Se interesa en el estudio de los campos magneticos primordiales. En el modelo cosmologio el universio deberia ser homogeneo e isotropico. Desde acá se forman las galaxias, lo cumulos y supercumulos. Una exisplacion a que hoy dia no sea homogeneo e isotropico puede ser los campos magneticos. Profesores: Leonardo Castañeda y -Juan manuel Tejeiro.

Aborda problemas como la generación y amplificaión de campos magnéticos conmológicos utilizando teoría de perturbaciones cosmologicas de alto orden, el estudio de las teorías extenidas de gravedad, especialmente aquellas de gravedad modificada f(R) y técnicas de reconstrucción de perfiles de masa de cúmulos de galaxias utilizando lentes gravitacionales.

\section{Temodinámica de agujeros negros }
Es un intento de terminar lo que inició hawking. La idea de buscar un objeto que una los mas pequeño (cuantica) y lo mas grande (gravitacion) desde aquí se llegó a los agujeros negros, es un escenario donde se pueden estudiar ambos fenomenos. Existen efectos cuanticos que cambian los fenomemos vistos en los agujeros negros, como emitir particulas tal como lo explicó hawking con la radiacion hawking, estas particulas se emiten en forma de un baño termico. 

Efectos cuánticos de la física del colapso gravitacional. con base en cascarones esfericos delgados de materia en colapso gravitacional, se describen los estados de vacío de los campos de materia en el contexto de la dinámica de campos térmicos. 

\hfill

\textbf{Proyectos }
\begin{itemize}
  \item Descripcion de la entropia bekenstein-hawking ocmo un efecto termico en temrinos del vacio cuantico de Boulware. 
  \item Descripcion geometrica de la entripía bekenstein- Hawking con base en el vacío del hartle-hawking. 
\end{itemize}

Como se trabaja con la relatividad esto se vuelve muy complicado ya que todo depende del observador. 

\section{Astrofísica de agujeros negros }

Se estudia los efectos que ocurren al rededor de los agujeros negros. 
\hfill

El trabajo en esta línea de investigación se enmarca principalmente en el estudio del movimiento de objetos astrofísicos rotantes que orbitan alrededor de agujeros negros y en el estudio y simulación de estructuras de acreción alredor. 

\section{Inteligencia artificial en astornomía }
Se utiliza el machine learning para el estudio de la astronomia como la evoluciond e las galaxias, estudio del sol, lentes gravitacionales, etc. profesores: Mario Armando Higuera, Santiago Vasgas Dominguez, Eduard Larrañaga. 

Uso de herramientas de inteligrancia artificial en la solución de problemas particulares de la astronomia y la astrofísica. 

\hfill

\textbf{Proyectos }
\begin{itemize}
  \item Generación de imágenes y características físicas (campo magnetico, perfil de velocidades, presión, etc.) en la atmosfera solar mediante redes neuronales. 
  \item Machine learning aplicado a la clasificación y descripción de núcleos activos de galaxias 
  \item Obtención de parámetris físicos de agujeros negros a partir de imagenes sinteticas y redes neuronares 
  \item Uso de PINNs
\end{itemize}

\section{Instrumentación Astronómica }
Se trabaja en cosas como la automatizacion de la cupula. Tambien en la construccion del primer inteferometro solar de ondas de radio de colombia con el cual se tienen una mejor resolucion de los datos gracias a la interferometria.

\end{document}
