\documentclass{article}

\usepackage[most]{tcolorbox}
\usepackage{physics}
\usepackage{graphicx}
\usepackage{float}
\usepackage{amsmath}
\usepackage{amssymb}


\usepackage[utf8]{inputenc}
\usepackage[a4paper, margin=1in]{geometry} % Controla los márgenes
\usepackage{titling}

\title{Caos \& Complejidad }
\author{Manuel Garcia.}
\date{\today}

\renewcommand{\maketitlehooka}{%
  \centering
  \vspace*{0.05cm} % Espacio vertical antes del título
}

\renewcommand{\maketitlehookd}{%
  \vspace*{2cm} % Espacio vertical después de la fecha
}

\newcommand{\caja}[3]{%
  \begin{tcolorbox}[colback=#1!5!white,colframe=#1!25!black,title=#2]
    #3
  \end{tcolorbox}%
}

\begin{document}
\maketitle

\section{Info del grupo }
Profesor: Carlos Viviescas. 

El grupo investiga en Informacion clasica y cuantica. El grupo se concentra mas en informacion y sistemas cuanticos. 

\section{Temas }
\begin{itemize}
  \item Metodos semiclasicos 
    \textbf{Caos cuantico }. Se hace mecanica cuantica en espacio de fase (ej. funcion de Wigner). 
  \item Aproximacion semiclasica a la teoria de perturbacion quantica. 
\end{itemize}


\section{Fundamentos de la mecanica cuantica }
\begin{itemize}
  \item Mediciones cuanticas en una configuracion unitaria. 
\end{itemize}

\section{Optica Cuantica }
\begin{itemize}
  \item QED en cavidades abiertas. Como cuantizar el campo electormagnetico en cavidades abiertas. 
  \item Propagacion de luz en sistemas disorientados. 
  \item Laseres aleatorios. 
\end{itemize}

\section{Termodinamica cuantica }
\begin{itemize}
  \item QT and unraveling selection - no quantum heat 
  \item Quantum fluctuation theorems and non-Markovianity.
\end{itemize}

\section{Informacion cuantica }
\begin{itemize}
  \item Dinamica de entrelazamiento (decoherencia). 
  \item Cuantificar entrelazamientos en estados aleatorios. 
  \item Entrelazamiento en estado de fase . 
  \item correlaciones cuanticas en particulas indistingibles. 
\end{itemize}

\section{Computacion Cuantica }
\begin{itemize}
  \item preparacion adiabatica de estados en un circuito QED. Siempre se busca preparar los estados de la forma mas perfecta pero esto toma muchisimo tiempo, en este tema se busca como acelerar este proceso de la preparacion de los estados. Se hizó un loop de feedback para ir prerando el estado. 
  \item Quantum Machine Learning. 
\end{itemize}

\section{Metrologia cuantica }
\begin{itemize}
  \item Caracterizacion de canales cuanticos gaussianos. 
  \item Peines de frecuencia. 
\end{itemize}

\section{Control cuantico, sistemas abiertos y decoherencia }
\begin{itemize}
  \item Trayectorial cuanticas y feedback. 
  \item Trajectorias cuanticas no markovianas. 
  \item Formalismo de reaccion coordinada y trayectorias cuanticas. 
  \item Decoherencia de gravedad cuantica. 
\end{itemize}

\section{Segunda charla }
\subsection{Bioinspiración }

\end{document}
