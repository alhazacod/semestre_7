\documentclass{article}

\usepackage[most]{tcolorbox}
\usepackage{physics}
\usepackage{graphicx}
\usepackage{float}
\usepackage{amsmath}
\usepackage{amssymb}


\usepackage[utf8]{inputenc}
\usepackage[a4paper, margin=1in]{geometry} % Controla los márgenes
\usepackage{titling}

\title{Clase 14 }
\author{Manuel Garcia.}
\date{\today}

\renewcommand{\maketitlehooka}{%
  \centering
  \vspace*{0.05cm} % Espacio vertical antes del título
}

\renewcommand{\maketitlehookd}{%
  \vspace*{2cm} % Espacio vertical después de la fecha
}

\newcommand{\caja}[3]{%
  \begin{tcolorbox}[colback=#1!5!white,colframe=#1!25!black,title=#2]
    #3
  \end{tcolorbox}%
}

\begin{document}
\maketitle

\section{Limites }
Otra forma de demostrar un limite sin hacerlo lo $ \epsilon $ y $ \delta $ es mostrar que por cualquier camino por el que nos vayamos llegamos al mismo valor.

\subsection{Propiedades de los límites en funciones complejas}
Sea $ f  $ y $ g  $ funciones complejas definidas en $ S  $ del plano complejo, y $ z_0  $ sea un punto de acumulacion de $ S  $. Si $ \underset{z \rightarrow z_0 }{lim }f(z)  \land \underset{z \rightarrow z_0 }{lim }g(z)   $ existen y las constantes $ c_1,c_2  $ son complejas, entonces: 
\begin{itemize}
  \item $ \underset{z \rightarrow z_0 }{lim }\left[c_1 f(z) + c_2 g(z) \right] = c_1 \underset{z \rightarrow z_0 }{lim }f(z) + c_2 \underset{z \rightarrow z_0 }{lim }g(z)  $.
  \item $ \underset{z  \rightarrow z_0 }{lim}f(x)g(x) = \underset{z \rightarrow z_0 }{lim}f(x) \underset{z  \rightarrow z_0 }{lim}g(z)  $
  \item $ \underset{z  \rightarrow z_0 }{lim}\frac{f(z) }{g(z) } = \displaystyle\frac{\underset{z  \rightarrow z_0 }{lim}f(z) }{\underset{z  \rightarrow z_0 }{lim}g(x)} \qquad$ si $ \underset{z  \rightarrow z_0 }{lim}g(z) \neq 0  $.
\end{itemize}

\textbf{Limites iterados }: Ejemplo: Encuentre que $ \underset{z  \rightarrow 0 }{lim}\displaystyle\frac{z }{\left|z \right|} $ no tiene limite.
\begin{align*}
  \underset{z  \rightarrow 0 }{lim}\displaystyle\frac{z }{\left|z \right|} &= \underset{\underset{y \rightarrow 0 }{x \rightarrow 0 }}{lim}\frac{x + i y }{\sqrt{x ^2 + y ^2} } \\
                                                                           &= \underset{\underset{y \rightarrow 0 }{x \rightarrow 0 }}{lim} \frac{x }{\sqrt{x ^2 + y ^2} } + i \underset{\underset{y \rightarrow 0 }{x \rightarrow 0 }}{lim} \frac{y }{\sqrt{x ^2+ y ^2} }\\
                                                                           &= i\\
\end{align*}
\begin{gather*}
                                                                           \text{Por otro lado: }\\
                                                                           \underset{z  \rightarrow 0 }{lim}\frac{z }{\left|z \right|} = \underset{\underset{x \rightarrow 0 }{y \rightarrow 0 }}{lim} \frac{x + i y }{\sqrt{x^2+y^2 } } = \underset{\underset{x \rightarrow 0 }{y \rightarrow 0 }}{lim} \frac{x }{\sqrt{x^2+y^2 } } + i \underset{\underset{x \rightarrow 0 }{y \rightarrow 0 }}{lim} \frac{y }{\sqrt{x^2+y^2 } } = 1
\end{gather*}

Ejemplo: Analizar la funcion $ f(z) = \left(\frac{z }{\bar z }\right)^2 $ cuando $ z \rightarrow 0  $.
\begin{align*}
  \left(\frac{x + i y }{x - i y }\right) ^2 &= \frac{(x ^2 - y ^2 + 2 ixy ) ^2}{(x ^2+ y ^2) ^2 + 4 x ^2 y ^2 }\\ 
    &= \frac{(x ^2- y ^2) ^2 + 4 ixy(x ^2- y ^2) - 4 x ^2y ^2}{(x ^2 - y ^2)^2 + 4 x ^2 y ^2 }\\
    &= \frac{(x ^2- y ^2)^2 - 4 x ^2y ^2 }{(x ^2 - y ^2)^2 + 4 x ^2 y ^2 } + i \frac{4xy(x ^2 - y ^2 )}{(x ^2 - y ^2)^2 + 4 x^2 y ^2}
\end{align*}

Desde $x$ el limite nos da 1 y desde $y$ nos da 1 pero esto no es suficiente para demostrar que el limite es 1. Ahora vamos a hacer $ y = mx  $ (porque nos acercamos al origen).

\subsection{Limites que involucran el infinito}
Para proceder al calculo se debe tener presente que $ z \rightarrow \infty $ en el contexto de los numeros complejos es equivalente a $ \left|z \right| \rightarrow  \infty $ y tambien $ f(z) \rightarrow \infty $ es $ \left|f(z) \right| \rightarrow \infty $. De esta forma: 
\begin{gather*}
  \underset{z  \rightarrow z_0 }{lim}f(z) = \infty \leftrightarrow \underset{z  \rightarrow z_0 }{lim}\left|f(z) \right| = \infty \\
  \underset{z  \rightarrow \infty}{lim}f(z) =L \leftrightarrow \underset{\left|z \right| \rightarrow \infty}{lim}\left|f(z)- L \right| = 0 \\
  \underset{z  \rightarrow \infty}{lim}f(z) = \infty \leftrightarrow \underset{\left|z \right| \rightarrow \infty}{lim}\left|f(z) \right| = 0 
\end{gather*}

Una propiedad importante similar al caso real: 
\begin{gather*}
  \underset{z  \rightarrow \infty}{lim}f(z) = L \leftrightarrow \underset{z  \rightarrow  0 }{lim}f(\frac{1}{z }) = L 
\end{gather*}
Ejemplo 1: 
\begin{gather*}
  \underset{z  \rightarrow \infty}{lim}\frac{1}{z } = \underset{z  \rightarrow 0 }{lim}\frac{1}{\frac{1}{z }} = \underset{z  \rightarrow 0 }{lim}z = 0  
\end{gather*}
Ejemplo 2: 
\begin{gather*}
  \underset{z  \rightarrow \infty}{lim}\frac{az + b }{cz + d } = \underset{z  \rightarrow \infty}{lim}\frac{a + \frac{b }{z }}{c + \frac{d }{z }} = \frac{a }{c } 
\end{gather*}
Ejemplo 3: 
\begin{align*}
  \underset{z  \rightarrow \infty}{lim}\quad e ^ {-z } &= \underset{z  \rightarrow 0 }{lim}\quad e ^ {- \frac{1}{x + i y }}\\ 
  &= \underset{z  \rightarrow 0 }{lim}\quad e ^ {- \frac{x + i y }{x ^2+ y ^2}}\\
  & = \underset{z  \rightarrow 0 }{lim}\quad e ^ {- \frac{x }{x ^2 + y ^2}} e ^ { \frac{i y }{x ^2 + y ^2 }}
\end{align*}
Al hacer el limite obtenemo sun valor diferente que al hacerlo por la derecha por lo que el limite no existe.

\end{document}
