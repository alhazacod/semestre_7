\documentclass{article}

\usepackage[most]{tcolorbox}
\usepackage{physics}
\usepackage{graphicx}
\usepackage{float}
\usepackage{amsmath}
\usepackage{amssymb}


\usepackage[utf8]{inputenc}
\usepackage[a4paper, margin=1in]{geometry} % Controla los márgenes
\usepackage{titling}

\title{Clase 18 }
\author{Manuel Garcia.}
\date{\today}

\renewcommand{\maketitlehooka}{%
  \centering
  \vspace*{0.05cm} % Espacio vertical antes del título
}

\renewcommand{\maketitlehookd}{%
  \vspace*{2cm} % Espacio vertical después de la fecha
}

\newcommand{\caja}[3]{%
  \begin{tcolorbox}[colback=#1!5!white,colframe=#1!25!black,title=#2]
    #3
  \end{tcolorbox}%
}

\begin{document}
\maketitle

\section{Teorema de Stokes }
Sea $ \omega \in \Omega ^ {r-1}(M) $ y $ c \in C_r(M)  $, entonces: 
\caja{red}{}{
  \begin{gather*}
    \displaystyle\int_{c }^{} d \omega \equiv \displaystyle\int_{\partial c }^{} \omega 
  \end{gather*}
}

\textbf{Metrica rimaniana: }
$ g  $ en una variedad $ M  $ es un tensor de tipo $ (0,2 ) $.
\begin{itemize}
  \item $g_p(U,V) = g_p(V,U)$
  \item $ g_p(U,U) \geq 0  $ donde la igualdad solo se cumple para $ U = 0  $
\end{itemize}
\textbf{Metrica pseudo-Riemanniana}
\begin{itemize}
  \item cumple la condicion 1 de la Riemanniana 
  \item Si $ g_p(U,V) = 0 \forall V \in T_p M  $ entonces $ U= 0  $
\end{itemize}

\end{document}
