\documentclass{article}

\usepackage[most]{tcolorbox}
\usepackage{physics}
\usepackage{graphicx}
\usepackage{float}
\usepackage{amsmath}
\usepackage{amssymb}


\usepackage[utf8]{inputenc}
\usepackage[a4paper, margin=1in]{geometry} % Controla los márgenes
\usepackage{titling}

\title{Clase 12 }
\author{Manuel Garcia.}
\date{\today}

\renewcommand{\maketitlehooka}{%
  \centering
  \vspace*{0.05cm} % Espacio vertical antes del título
}

\renewcommand{\maketitlehookd}{%
  \vspace*{2cm} % Espacio vertical después de la fecha
}

\newcommand{\caja}[3]{%
  \begin{tcolorbox}[colback=#1!5!white,colframe=#1!25!black,title=#2]
    #3
  \end{tcolorbox}%
}

\begin{document}
\maketitle

\section{Funcion Logaritmo }
\caja{red}{}{
  \begin{gather*}
    \omega = \log{z } \leftrightarrow z =  e ^ {\omega}\\
    \log_\alpha {z} = \log{\left|z \right|} + (\theta + 2 \pi k )i , \qquad k = ...,-2,-1,0,1,2,...\\
    \log_\alpha z: \text{Rama }\alpha \text{ del log de z }\\
    \alpha < Arg(z) \leq \alpha+ 2 \pi\\
  \end{gather*}
}

\textbf{Ej } $ \log_{\frac{45 \pi}{7 }} (\sqrt{3 } - i  ) = ? $
\begin{gather*}
  \left|z \right| = \sqrt{(\sqrt{3 } )^2 + (-1)^2 } = 2\\ 
  \tan{\theta} = \frac{1}{\sqrt{3 } }\frac{\sqrt{3 } }{\sqrt{3 } }\\
  \theta = -\tan ^ {-1 }{\frac{\sqrt{3 } }{3 }} = \frac{- \pi}{6 }\\
  \alpha = \frac{45 \pi }{7 }\\
  \alpha + 2 \pi = \frac{45 \pi}{7 }+ 2 \pi = \frac{59 \pi}{7 }\\
  \frac{45 \pi }{7 } < Arg(z) \leq \frac{59 \pi }{7 }\\
  Arg(z) = \frac{45 \pi }{7 } + \frac{59\pi }{42 } = \frac{329 \pi }{42} \qquad \frac{59\pi }{42 }\text{ es la diferencia entre }\frac{45\pi}{7} \text{ y } \frac{-\pi}{6 }. \text{Reducido es }\frac{11 \pi }{6 }- \frac{3 \pi }{7 } = \frac{59 \pi}{42 }\\
  \frac{45\pi}{7 } - \underset{\# \text{ vueltas }}{3 }(2\pi ) = 3 \frac{3 \pi}{7 }
\end{gather*}

\textbf{Ej }$ \log_ \frac{75 \pi }{8 }\frac{(-1 -\sqrt{3 }i  )}{4 }$
\begin{gather*}
  \left|z \right| = \frac{1}{2} \qquad \theta = - \frac{2\pi}{3 }\\
  \frac{75 \pi}{8 }< Arg(z) \leq \frac{75\pi }{8 }+ 2\pi= \frac{9\pi}{8 }\\
  \frac{75 \pi}{8 } \text{ da 4 vueltas } \frac{75 \pi}{8 } - 4(2\pi)= \frac{11\pi}{8 } \text{ entonces } \frac{11\pi}{8 }- \frac{4\pi}{3 } - 2\pi = \frac{47 \pi}{24 }\\
  Arg(z) = \frac{75 \pi }{8 }+ \frac{47 \pi}{24 } = \frac{34 \pi}{3}
\end{gather*}

\section{Potencias de complejos }
En general $ \log{z_1z_2} \neq \log{z_1}+\log{z_2} $ para los complejos.

\caja{black}{}{
  \begin{gather*}
    z ^ {a } = e ^ {a \log z }, \quad z \neq 0 
  \end{gather*}
  Como el logaritmo tiene muchas raices esta ecuacion es una ecuacion multivaluada.
}

\textbf{Ej }Vamos a calcular $ i ^ {i+1 }$.
\begin{align*}
  z = i, \quad a = i+1 \quad \rightarrow \quad i ^ {i +1 } &= e ^ {(i+ 1) \log{i}}  = e ^ {(i+ 1) \log{e ^ {i(\frac{\pi}{2} + 2 \pi k )}}}\\
  i ^ {i + 1 } &= e ^ {(i+1 )\left(i \left(\frac{\pi }{2} + 2\pi k \right)\right)}\\
               & = e ^ {- \left(\frac{\pi}{2} + 2 \pi k \right)} e ^ {i \left(\frac{\pi}{2} + 2 \pi  k \right)}\\
               & = e ^ {- \left(\frac{\pi}{2} + 2 \pi k \right)}\{\cos{\frac{\pi }{2} + 2 \pi k } + i \sin{\frac{\pi }{2} + 2 \pi k } \}\\
  \text{(Solo el V.P) } &= e ^ {- \frac{\pi}{2}}\{i \} = i e ^ {- \frac{\pi}{2}}
\end{align*}

\textbf{Ej } encontrar $ \left(\frac{i + 1 }{i - 1 }\right)^ {i + 3 }$. 
\begin{gather*}
  \frac{i + 1 }{i - 1 }\frac{i + 1 }{i + 1 } = \frac{2 i }{- 2 } = -i   \text{Entonces: }\left(\frac{i + 1 }{i - 1 }\right)^ {i + 3 }  = (-i ) ^ {i + 3 }\\
  - i ^ {i + 3 } = e ^ {(i + 3 ) log(-i )}\\
  \text{Calculamos } \log {-i} = \log{i} + i \left(- \frac{\pi}{2} + 2 \pi k \right)\\
  \text{Entonces: }\\
  e ^ {(i + 3 ) i \left(\frac{\pi }{2} + 2 \pi k \right)} = e ^ {(- 1 + 3 i ) \left(-\frac{\pi}{2} + 2 k \pi \right)} = e ^ {\left(\frac{\pi}{2} - 2k \pi\right)}\left(\cos{- \frac{3 \pi }{2} + 6 \pi k } + i \sin{- \frac{3 \pi }{2 } + 6 \pi k }\right) 
\end{gather*}

\textbf{Ej } $ i ^ {i ^ {i }} = ?  $.
\begin{gather*}
  i ^ {i ^ {i }} = e ^ {i ^ {i } \log{i }} = e ^ { i ^ {i }i \frac{\pi}{2}}= e ^ {i ^ {i+1 } \frac{\pi}{2}} \\
  \text{Teniamos que } i ^ {i + 1 } = i e ^ {-\pi/2} \text{ entonces: }\\
  i ^ {i ^ {i }} = e ^ {i e ^ {- \frac{\pi }{2}}\frac{\pi}{2}} = \left(e ^ {i \frac{\pi}{2}}\right)^ {e ^ {- \frac{\pi}{2}}} =  i ^ {e ^ {- \frac{\pi}{2}}}
\end{gather*}

\textbf{Demostrar } que $ \sin ^ {-1 }{z } = - i \log{\left(i z + (1 - z ^2)^ {1/2 }\right)} $
\begin{gather*}
  \text{En nuestra definicion: } \sin{z } = \frac{e ^ {i z }- e ^ {-i z }}{2 i } \quad \rightarrow \quad \sin{z } = \frac{e ^ {iz }- \frac{1}{e ^ {iz }}}{2i } = \frac{e ^ {2iz } - 1 }{2i e ^ {iz }}\\
  \text{Queremos sustituir }z \rightarrow \sin ^ {-1 }z \\
  \sin{\sin ^ {-1 } z } = \frac{e ^ {2 i \sin ^ {-1 }z }- 1 }{2i e ^ {i \sin ^ {-1 }z }}\\
  \text{Entonces tenemos que } z = \frac{e ^ {2 i \sin ^ {-1 }z }- 1 }{2 i e ^ {i \sin ^ {-1 }z }}\\
  \text{Despejamos }\sin ^ {-1 }z \\
  2 i z e ^ {i \sin ^ {-1 } z } = e ^ {2i \sin ^ {-1 } z }- 1\\
  0 = e ^ {2i \sin ^ {-1 }z }- 2 i z e ^ {i \sin ^ {-1 } z }-1 \\
  e ^ {i \sin ^ {-1 }z } = \frac{2 i z \pm \sqrt{-4 z ^2 + 4 } }{2} = iz \pm  \sqrt{1 - z ^2} \\
  i \sin ^ {-1 } z = \log{\left(i z \pm \sqrt{1 - z ^2 } \right)}\\
  \sin ^ {-1 }z =  - i \log{\left(i z \pm \sqrt{1 - z ^2} \right)}
\end{gather*}

\caja{blue}{Ejercicios }{
  \begin{itemize}
    \item $ \sin ^ {-1 }i = ?  $. 
    \item $ \sin ^ {-1 } (-i )  = ? $
    \item $\tan ^ {-1 }(z) = ? $
    \item $\cos ^ {-1 }z = ?$
  \end{itemize}
}

\end{document}
