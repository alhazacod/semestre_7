\documentclass{article}

\usepackage[most]{tcolorbox}
\usepackage{physics}
\usepackage{graphicx}
\usepackage{amsmath}
\usepackage{amssymb}


\usepackage[utf8]{inputenc}
\usepackage[a4paper, margin=1in]{geometry} % Controla los márgenes
\usepackage{titling}

\title{Clase 1 }
\author{Manuel Garcia.}
\date{\today}

\renewcommand{\maketitlehooka}{%
  \centering
  \vspace*{0.05cm} % Espacio vertical antes del título
}

\renewcommand{\maketitlehookd}{%
  \vspace*{2cm} % Espacio vertical después de la fecha
}

\newcommand{\caja}[3]{%
  \begin{tcolorbox}[colback=#1!5!white,colframe=#1!25!black,title=#2]
    #3
  \end{tcolorbox}%
}

\begin{document}
\maketitle

\section{Campo (cuerpo) de numeros complejos}
\caja{green}{Numero complejo y operaciones}{
  \begin{gather}
    z = x+iy 
    \label{eq:numero_complejo }
  \end{gather}
  \begin{itemize}
    \item suma: $z_1+z_2  = x_1+iy_1+(x_2+iy_2) =  x_1+x_2+i(y_1+y_2)$
    \item producto: $ z_1 * z_2 = (x_1x_2-y_1y_2)+i(x_1y_2+x_2y_1) $
  \end{itemize}
}
\caja{green}{Propiedades }{
  \begin{itemize}
    \item Cerradura $ z_1+z_2 = x_1+x_2 +i(y_1+y_2) \in \mathbb{C} $
    \item Asopciatividad $ z_1+(z_2+z_3) = x_1+iy_1+x_2+iy_2+x_3+iy_3 = (z_1+z_2)+z_3 $
    \item conmutativa: $ z_1+z_2 = x_1+iy_1+x_2+iy_2 = x_2+iy_2+x_1+iy_1 $
    \item existencia elemento neutro 
      \begin{gather}
         z_1+0 = x_1+iy_1+0+i0\\
         = (x_1+0)+i(y_1+0) = x_1+iy_1 = z_1
      \end{gather}
    \item existencia inverso aditivo
      \begin{gather}
         z_1+(-z_1) = (x_1+iy_1) + (-x_1-iy_1) = (x_1+x_2)+i(y_1+y_1) = 0 +i0
      \end{gather}
    \item Asociatividad del producto $ z * (z_2*z_3) = (z_1*z_2)z_3 $
    \item elemento neutro del producto $ z_1*1 = z_1 $
  \end{itemize}
}


\end{document}
